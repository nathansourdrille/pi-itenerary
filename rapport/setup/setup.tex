\documentclass[12pt,a4paper,openright]{report}

% Encodage et langue
\usepackage[utf8]{inputenc} % Encodage UTF-8
\usepackage[T1]{fontenc} % Encodage des fontes
\usepackage[french,english]{babel} % Langues : français et anglais

% Police et espacement
\usepackage[sc]{mathpazo} % Police Palatino
\linespread{1.05} % Espacement des lignes

% Couleurs et graphiques
\usepackage{xcolor} % Gestion des couleurs
\definecolor{customblue}{RGB}{33,26,82} % Définition de la couleur aaublue
\usepackage{graphicx} % Inclusion d'images
\usepackage{wrapfig} % Figures et tableaux enroulés dans le texte
\usepackage{float} % Interface améliorée pour les objets flottants
\usepackage{caption} % Personnalisation des légendes
\captionsetup{%
  font=footnotesize, % Taille de police des légendes
  labelfont=bf % Police en gras pour les étiquettes (par ex., Figure 3.2)
}

% Mise en page et formatage
\usepackage[left=2cm, right=2cm, top=2cm, bottom=2cm]{geometry} % Marges du document
\usepackage{fancyhdr} % En-têtes et pieds de page personnalisés
\fancyhf{} % Supprimer tous les champs d'en-tête et de pied de page
\renewcommand{\headrulewidth}{0pt} % Supprimer la ligne horizontale dans l'en-tête
\fancyhead[RE]{\small\nouppercase\leftmark} % Page paire - titre du chapitre
\fancyhead[LO]{\small\nouppercase\rightmark} % Page impaire - titre de la section
\fancyhead[LE,RO]{\thepage} % Numéro de page sur toutes les pages
\usepackage{lastpage} % Référence au nombre de pages du document
\pagestyle{fancy}
\raggedbottom % Ne pas étirer le contenu de la page, insérer de l'espace blanc en bas de la page

% Environnements mathématiques et théorèmes
\usepackage{amsmath, amsthm, amssymb, amsfonts} % Paquets AMS pour les maths
\usepackage[framed,amsmath,thmmarks]{ntheorem} % Environnements pour les théorèmes

% Autres utilitaires
\usepackage{array,booktabs} % Gestion améliorée des tableaux
\usepackage{framed} % Encadrés pour les parties importantes
\usepackage{listings} % Formatage et mise en évidence du code source
\usepackage{courier} % Police Courier pour les listings
\usepackage{changepage} % Ajuster les marges pour des parties spécifiques de la page
\usepackage{multicol} % Colonnes multiples
\usepackage{hyperref} % Hyperliens dans le PDF
\hypersetup{%
	pdfpagelabels=true,
	plainpages=false,
	pdfauthor={Author(s)},
	pdftitle={Title},
	pdfsubject={Subject},
	bookmarksnumbered=true,
	colorlinks=true,
    breaklinks=true,
    linkcolor=black,
    citecolor=blue,
    filecolor=magenta,
    urlcolor=blue,
    linkbordercolor={1 1 1},
	pdfstartview=FitH
}

% Définition des couleurs
\definecolor{hellgelb}{rgb}{1,1,0.8}
\definecolor{colKeys}{rgb}{0,0,1}
\definecolor{colIdentifier}{rgb}{0,0,0}
\definecolor{colComments}{rgb}{0,0.5,0}
\definecolor{colString}{rgb}{0.62,0.12,0.94}
\definecolor{INSA_GM}{cmyk}{0.6,0,0,0}
\definecolor{INSA_GRIS}{cmyk}{0.7,0.6,0.5,0.3}
\definecolor{INSA_BLEU}{cmyk}{1,0.9,0.1,0}

% Commandes personnalisées
\newcommand{\insertrefproj}[1]{}
\newcommand{\refproj}[1]{\renewcommand{\insertrefproj}{\textbf{\color{INSA_GRIS}#1}}}

% Styles de page fancy
\pagestyle{fancy}

\fancypagestyle{courant}{
\fancyhf{}
\setlength{\headheight}{27pt}
\fancyhead[L]{\raisebox{-2mm}{\includegraphics[width=30mm]{Images/Logo INSA.png}}}
\fancyhead[C]{}
\fancyhead[R]{\color{INSA_GRIS}\thepage}
\fancyfoot[L]{\insertrefproj}
\fancyfoot[R]{}
\renewcommand{\headrulewidth}{0pt}
\renewcommand{\footrulewidth}{0.2pt}
}

\fancypagestyle{special}{%
\pagestyle{courant}
\fancyfoot{}
\renewcommand{\footrulewidth}{0pt}
}

\fancypagestyle{plain}{%
\fancyhf{}%
\pagestyle{courant}
}

% Formatage des titres et sections
\usepackage{titlesec}
\titleformat{\chapter}[display]{\normalfont\huge\bfseries}{\chaptertitlename\ \thechapter}{20pt}{\Huge}
\titleformat*{\section}{\normalfont\Large\bfseries}
\titleformat*{\subsection}{\normalfont\large\bfseries}
\titleformat*{\subsubsection}{\normalfont\normalsize\bfseries}

% Autres configurations et paquets
\setlength{\parindent}{0pt} % Pas d'indentation pour les paragraphes
\usepackage[contents={},color=gray]{background} % Couleur de fond
\usepackage{epic,eepic} % Positionnement pour l'image de couverture

% Réduire les espacements des titres de chapitre, trop importants par défaut
\titlespacing*{\chapter}{0pt}{-25pt}{15pt} % {alinéa}{au-dessus}{en-dessous}

% Espacement des insertions de figures
\setlength{\textfloatsep}{7pt} % Espace entre le texte et une figure flottante
\setlength{\intextsep}{7pt} % Espace entre le texte et une figure insérée
\setlength{\floatsep}{7pt} % Espace entre deux flottants
\renewcommand{\thefigure}{\arabic{figure}}
\newcommand{\figcaptionwithsource}[3]{\caption[#1\newline #2]{#1} \addtocontents{lof}{\protect\vspace{1\baselineskip}}}
\renewcommand{\listfigurename}{}
\counterwithout{figure}{chapter} % Ne pas réinitialiser le numéro des figures à chaque chapitre

\newcommand{\annexeref}[1]{Voir Annexe \ref{#1}}

% Page vide sans numéro de page
\let\origdoublepage\cleardoublepage
\newcommand{\clearemptydoublepage}{%
  \clearpage
  {\pagestyle{empty}\origdoublepage}%
}
\let\cleardoublepage\clearemptydoublepage

% Supprimer le numéro de page d'une page
\usepackage{nopageno}

% Bibliographie et index
\usepackage[nottoc]{tocbibind}

% Notes à faire
\usepackage[
  colorinlistoftodos,
  textwidth=\marginparwidth, 
  textsize=scriptsize,
]{todonotes}
