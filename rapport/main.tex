\documentclass[12pt,a4paper,openright]{report}

% Encodage et langue
\usepackage[utf8]{inputenc} % Encodage UTF-8
\usepackage[T1]{fontenc} % Encodage des fontes
\usepackage[french,english]{babel} % Langues : français et anglais

% Police et espacement
\usepackage[sc]{mathpazo} % Police Palatino
\linespread{1.05} % Espacement des lignes

% Couleurs et graphiques
\usepackage{xcolor} % Gestion des couleurs
\definecolor{customblue}{RGB}{33,26,82} % Définition de la couleur aaublue
\usepackage{graphicx} % Inclusion d'images
\usepackage{wrapfig} % Figures et tableaux enroulés dans le texte
\usepackage{float} % Interface améliorée pour les objets flottants
\usepackage{caption} % Personnalisation des légendes
\captionsetup{%
  font=footnotesize, % Taille de police des légendes
  labelfont=bf % Police en gras pour les étiquettes (par ex., Figure 3.2)
}

% Mise en page et formatage
\usepackage[left=2cm, right=2cm, top=2cm, bottom=2cm]{geometry} % Marges du document
\usepackage{fancyhdr} % En-têtes et pieds de page personnalisés
\fancyhf{} % Supprimer tous les champs d'en-tête et de pied de page
\renewcommand{\headrulewidth}{0pt} % Supprimer la ligne horizontale dans l'en-tête
\fancyhead[RE]{\small\nouppercase\leftmark} % Page paire - titre du chapitre
\fancyhead[LO]{\small\nouppercase\rightmark} % Page impaire - titre de la section
\fancyhead[LE,RO]{\thepage} % Numéro de page sur toutes les pages
\usepackage{lastpage} % Référence au nombre de pages du document
\pagestyle{fancy}
\raggedbottom % Ne pas étirer le contenu de la page, insérer de l'espace blanc en bas de la page

% Environnements mathématiques et théorèmes
\usepackage{amsmath, amsthm, amssymb, amsfonts} % Paquets AMS pour les maths
\usepackage[framed,amsmath,thmmarks]{ntheorem} % Environnements pour les théorèmes

% Autres utilitaires
\usepackage{array,booktabs} % Gestion améliorée des tableaux
\usepackage{framed} % Encadrés pour les parties importantes
\usepackage{listings} % Formatage et mise en évidence du code source
\usepackage{courier} % Police Courier pour les listings
\usepackage{changepage} % Ajuster les marges pour des parties spécifiques de la page
\usepackage{multicol} % Colonnes multiples
\usepackage{hyperref} % Hyperliens dans le PDF
\hypersetup{%
	pdfpagelabels=true,
	plainpages=false,
	pdfauthor={Author(s)},
	pdftitle={Title},
	pdfsubject={Subject},
	bookmarksnumbered=true,
	colorlinks=true,
    breaklinks=true,
    linkcolor=black,
    citecolor=blue,
    filecolor=magenta,
    urlcolor=blue,
    linkbordercolor={1 1 1},
	pdfstartview=FitH
}

% Définition des couleurs
\definecolor{hellgelb}{rgb}{1,1,0.8}
\definecolor{colKeys}{rgb}{0,0,1}
\definecolor{colIdentifier}{rgb}{0,0,0}
\definecolor{colComments}{rgb}{0,0.5,0}
\definecolor{colString}{rgb}{0.62,0.12,0.94}
\definecolor{INSA_GM}{cmyk}{0.6,0,0,0}
\definecolor{INSA_GRIS}{cmyk}{0.7,0.6,0.5,0.3}
\definecolor{INSA_BLEU}{cmyk}{1,0.9,0.1,0}

% Commandes personnalisées
\newcommand{\insertrefproj}[1]{}
\newcommand{\refproj}[1]{\renewcommand{\insertrefproj}{\textbf{\color{INSA_GRIS}#1}}}

% Styles de page fancy
\pagestyle{fancy}

\fancypagestyle{courant}{
\fancyhf{}
\setlength{\headheight}{27pt}
\fancyhead[L]{\raisebox{-2mm}{\includegraphics[width=30mm]{Images/Logo INSA.png}}}
\fancyhead[C]{}
\fancyhead[R]{\color{INSA_GRIS}\thepage}
\fancyfoot[L]{\insertrefproj}
\fancyfoot[R]{}
\renewcommand{\headrulewidth}{0pt}
\renewcommand{\footrulewidth}{0.2pt}
}

\fancypagestyle{special}{%
\pagestyle{courant}
\fancyfoot{}
\renewcommand{\footrulewidth}{0pt}
}

\fancypagestyle{plain}{%
\fancyhf{}%
\pagestyle{courant}
}

% Formatage des titres et sections
\usepackage{titlesec}
\titleformat{\chapter}[display]{\normalfont\huge\bfseries}{\chaptertitlename\ \thechapter}{20pt}{\Huge}
\titleformat*{\section}{\normalfont\Large\bfseries}
\titleformat*{\subsection}{\normalfont\large\bfseries}
\titleformat*{\subsubsection}{\normalfont\normalsize\bfseries}

% Autres configurations et paquets
\setlength{\parindent}{0pt} % Pas d'indentation pour les paragraphes
\usepackage[contents={},color=gray]{background} % Couleur de fond
\usepackage{epic,eepic} % Positionnement pour l'image de couverture

% Réduire les espacements des titres de chapitre, trop importants par défaut
\titlespacing*{\chapter}{0pt}{-25pt}{15pt} % {alinéa}{au-dessus}{en-dessous}

% Espacement des insertions de figures
\setlength{\textfloatsep}{7pt} % Espace entre le texte et une figure flottante
\setlength{\intextsep}{7pt} % Espace entre le texte et une figure insérée
\setlength{\floatsep}{7pt} % Espace entre deux flottants
\renewcommand{\thefigure}{\arabic{figure}}
\newcommand{\figcaptionwithsource}[3]{\caption[#1\newline #2]{#1} \addtocontents{lof}{\protect\vspace{1\baselineskip}}}
\renewcommand{\listfigurename}{}
\counterwithout{figure}{chapter} % Ne pas réinitialiser le numéro des figures à chaque chapitre

\newcommand{\annexeref}[1]{Voir Annexe \ref{#1}}

% Page vide sans numéro de page
\let\origdoublepage\cleardoublepage
\newcommand{\clearemptydoublepage}{%
  \clearpage
  {\pagestyle{empty}\origdoublepage}%
}
\let\cleardoublepage\clearemptydoublepage

% Supprimer le numéro de page d'une page
\usepackage{nopageno}

% Bibliographie et index
\usepackage[nottoc]{tocbibind}

% Notes à faire
\usepackage[
  colorinlistoftodos,
  textwidth=\marginparwidth, 
  textsize=scriptsize,
]{todonotes}

% Load silence package to suppress warnings
\usepackage{silence}
% Filter out specific warnings related to everypage
\WarningFilter{everypage}{You appear to be running a version of LaTeX providing the new functionality}


% see, e.g., http://en.wikibooks.org/wiki/LaTeX/Customizing_LaTeX#New_commands
% for more information on how to create macros

%%%%%%%%%%%%%%%%%%%%%%%%%%%%%%%%%%%%%%%%%%%%%%%%
% Macros for the titlepage
%%%%%%%%%%%%%%%%%%%%%%%%%%%%%%%%%%%%%%%%%%%%%%%%
%Creates the aau titlepage
\newcommand{\customtitlepage}[3]{%
  {
    %set up various length
    \ifx\titlepageleftcolumnwidth\undefined
      \newlength{\titlepageleftcolumnwidth}
      \newlength{\titlepagerightcolumnwidth}
    \fi
    \setlength{\titlepageleftcolumnwidth}{0.5\textwidth-\tabcolsep}
    \setlength{\titlepagerightcolumnwidth}{\textwidth-2\tabcolsep-\titlepageleftcolumnwidth}
    %create title page
    \thispagestyle{empty}
    \noindent%
    \begin{tabular}{@{}ll@{}}
      \parbox{\titlepageleftcolumnwidth}{
          \includegraphics[width=\titlepageleftcolumnwidth]{Images/Logo INSA.png}
        }
      } &
      \parbox{\titlepagerightcolumnwidth}{\raggedleft\sf\small
        #2
      }\bigskip\\
       #1 &
      \parbox[t]{\titlepagerightcolumnwidth}{%
      \iflanguage{french}{%
        \textbf{Résumé:}\bigskip\par
        }{%
        \textbf{Abstract:}\bigskip\par
        }
        \fbox{\parbox{\titlepagerightcolumnwidth-2\fboxsep-2\fboxrule}{%
          #3
        }}
      }\\
    \end{tabular}
    \vfill
    \iflanguage{french}{%
      \noindent{\footnotesize\emph{Le contenu de ce rapport est librement disponible, mais sa publication (avec référence) ne peut être entreprise qu'avec l'accord de l'auteur.}}
    }{%
      \noindent{\footnotesize\emph{The content of this report is freely available, but publication (with reference) may only be pursued due to agreement with the author.}}
    }
    \clearpage
  }



%Create english project info
\newcommand{\englishprojectinfo}[7]{%
  \parbox[t]{\titlepageleftcolumnwidth}{
    \textbf{Title:}\\ #1\bigskip\par
    \textbf{Subjects:}\\ #2\bigskip\par
    \textbf{Project Period:}\\ #3\bigskip\par
    \textbf{Project Group:}\\ #4\bigskip\par
    \textbf{Participant(s):}\\ #5\bigskip\par
    \textbf{Supervisor(s):}\\ #6\bigskip\par
    \textbf{Page Numbers:} \pageref{LastPage}\bigskip\par
    \textbf{Date of Completion:}\\ #7
  }
}

\newcommand{\frenchprojectinfo}[7]{%
  \parbox[t]{\titlepageleftcolumnwidth}{
    \textbf{Titre :}\\ #1\bigskip\par
    \textbf{Matières :}\\ #2\bigskip\par
    \textbf{Période du projet :}\\ #3\bigskip\par
    \textbf{Groupe de projet :}\\ #4\bigskip\par
    \textbf{Participant(s) :}\\ #5\bigskip\par
    \textbf{Superviseur(s) :}\\ #6\bigskip\par
    \textbf{Nombre de pages :} \pageref{LastPage}\bigskip\par
    \textbf{Date de réalisation :}\\ #7
  }
}

%%%%%%%%%%%%%%%%%%%%%%%%%%%%%%%%%%%%%%%%%%%%%%%%
% An example environment
%%%%%%%%%%%%%%%%%%%%%%%%%%%%%%%%%%%%%%%%%%%%%%%%
\theoremheaderfont{\normalfont\bfseries}
\theorembodyfont{\normalfont}
\theoremstyle{break}
\def\theoremframecommand{{\color{gray!50}\vrule width 5pt \hspace{5pt}}}
\newshadedtheorem{exa}{Example}[chapter]
\newenvironment{example}[1]{%
		\begin{exa}[#1]
}{%
		\end{exa}
}

\usepackage[final]{pdfpages} 

% Fix for ntheorem conflict
\makeatletter
\let\c@theorem\relax
\makeatother

% Fix for todonotes margin warning
\setlength{\marginparwidth}{2cm}

\title{MyTemplate}
\refproj{Projet Intégratif ITInéraire}
\author{Paul Fougeray}
\date{Avril 2025}


\begin{document}

\pdfbookmark[0]{Front page}{label:frontpage}%

\begin{titlepage}
\newgeometry{top=0cm,bottom=1.2cm,right=0cm,left=0cm}

  \backgroundsetup{
   scale=1.1,
   angle=0,
   opacity=1.0,  %% adjust
   contents={\includegraphics[width=\paperwidth,height=\paperheight]{Images/waves.pdf}}
    }
		
  \begin{center} %%please do not change the height or width of the frontpage image
    \centerline{\includegraphics[totalheight=0.5\paperwidth,width=1\paperwidth]{Images/coverpageImage.jpg}}% 
  \end{center}
	
	\vspace*{-1.1cm}
  {\noindent\color{customblue}\fboxsep0pt\colorbox{white}{\begin{tabular}{@{}p{\paperwidth}@{}}
    \centerline{
    \begin{minipage}{0.85\textwidth}
        \bigskip
				\bigskip
        \centering
        \Huge{\textbf{
          Projet Intégratif ITIneraire
        }}
    \end{minipage}
    }
		

			
	\centerline{
	\begin{minipage}{0.9\textwidth}
        \bigskip
        \centering
        {\Large
        Arrigoni Ambroise, Fougeray Paul, Sanson Dylan, Sourdrille Nathan, Zouaghi Rayan
        }
    \end{minipage}
    }
			
			
    \centerline{
    \begin{minipage}{0.9\textwidth}
        \bigskip
        \centering
%% Comment this section if you are not doing Bachelor or Master Project   
        {\Large
        ITI3 groupe 1
      %Bachelor Project
        }
        \smallskip
    \end{minipage}
    }
			
  \end{tabular}}}

  \vfill
  \begin{figure}[!b]
	\centering
    \includegraphics[width=0.2\paperwidth]{Images/Logo INSA.png}
  \end{figure}
\end{titlepage}
\restoregeometry
\thispagestyle{empty}
{\small
\strut\vfill % push the content to the bottom of the page
\noindent Copyright \copyright{} Exemple 2024\par
% \vspace{0.2cm}
% \noindent Lorem ipsum dolor sit amet, consectetur adipiscing elit. Nullam sit amet interdum tortor. Phasellus nec metus est. Nullam lacinia arcu blandit, ornare sapien sed, hendrerit leo. Cras placerat turpis id leo bibendum facilisis. Praesent non nunc ut tellus laoreet egestas tempor vel sem. Cras rhoncus purus ac aliquam auctor. Donec lacinia laoreet nisl, sit amet porttitor mauris fringilla sed. Proin volutpat varius tempor. Sed dictum risus lacinia mi bibendum, ut porttitor metus ornare. Sed mollis dolor mauris, sit amet sodales nunc venenatis varius. Vestibulum ut dolor venenatis eros imperdiet dictum.
}
\clearpage

\cleardoublepage
{\selectlanguage{french}
\pdfbookmark[0]{French title page}{label:titlepage_fr}
\customtitlepage{%
  \frenchprojectinfo{
    Projet Intégratif ITIneraire %title
  }{%
    Capteurs et Statistiques %theme
  }{%
    Mars-Mai 2025 %project period
  }{%
    Groupe 9 % project group
  }{%
    %list of group members
    Arrigoni Ambroise\\
    Fougeray Paul\\
    Sanson Dylan\\
    Sourdrille Nathan\\
    Zouaghi Rayan
  }{%
    %list of supervisors
    Condat Robin\\
    Rogozan Alexandrina
  }{%
    \today % date of completion
  }%
}{%department and address
  \textbf{Département ITI}\\
  INSA Rouen Normandie\\
  \href{https://www.insa-rouen.fr/}{https://www.insa-rouen.fr/}
}{% Résumé
    L'objectif de ce projet d'intégration est de concurrencer l'application mobile Strava,
    dédiée pour l'enregistrement des activités sportives par GPS. Pour cela, vous devrez concevoir
    et développer un système d'acquisition permettant l'estimation de trajectoire d'un parcours
    fait à pied.
}
}


% À partir d'ici, le style courant sera appliqué à toutes les pages
\pagestyle{courant}

% Table des matières
\tableofcontents
% Hiérarchie
\setcounter{page}{1}
% Introduction
\chapter{Introduction}

Ce document présente un exemple d'utilisation de plusieurs fonctionnalités en LaTeX.


\chapter{Présentation des capteurs à notre disposition}
Cette partie a pour but de présenter les capteurs que nous avons utilisés lors des travaux pratiques.

\section{Description et fonctionnement}

\subsection{GPS}
Le capteur GPS (Global Positioning System) est un composant permettant de capter les signaux envoyés par les satellites GPS afin de déterminer une position géographique précise. Pour rappel, le GPS est le nom du système GNSS (Global Navigation Satellite System) développé par les États-Unis. Il s'agit d’un système de positionnement basé sur des satellites artificiels placés en orbite et fonctionnant de manière coordonnée.
\vspace{0.5cm}\\
Ce réseau repose sur une constellation d’environ 30 satellites opérationnels, situés à une altitude d’environ 20 200 km. La précision pour les usages civils varie généralement entre 3 et 5 mètres. Plus le nombre de satellites captés est élevé, plus la précision augmente, notamment grâce à la correction des erreurs liées à l’atmosphère ou aux interférences.
\vspace{0.5cm}\\
Il fournit typiquement des données de latitude, longitude, altitude, vitesse et temps. Ce capteur est essentiel pour le suivi de trajectoire.

\subsection{Accéléromètre}
Un accéléromètre est un dispositif qui mesure l'accélération (ou le mouvement) auquel il est soumis en appliquant le principe fondamentale de la dynamique: \\
\begin{adjustwidth}{7.5cm}{0cm} $\vec{F} = m \vec{a}$.
\end{adjustwidth}
\vspace{0.2cm}\\
En termes simples, l'accélèromètre détecte les changements de vitesse ou de direction. L'accéléromètre fonctionne en mesurant la force exercée sur un petit composant interne, souvent une masse suspendue à un ressort ou un capteur piézoélectrique. Cette masse se déplace en fonction des variations d'accélération de l'objet. Plus l'objet accélère, plus cette masse se déplace. Ce mouvement est ensuite mesuré par le capteur.
\vspace{0.5cm}\\
L'accéléromètre mesure les accélérations linéaires le long des trois axes (X, Y et Z). Il permet ainsi de détecter les mouvements, les chocs, et d’estimer l’orientation d’un objet (comme l'inclinaison).

\subsection{Gyroscope}
Le gyroscope est un capteur qui mesure la vitesse angulaire, c’est-à-dire la rapidité avec laquelle un objet tourne autour de ses axes (X, Y ou Z). Contrairement à l’accéléromètre qui détecte des mouvements linéaires, le gyroscope mesure les rotations. Le principe de fonctionnement du gyroscope repose sur la conservation du moment cinétique. Lorsqu’un rotor (ou masse tournante) est mis en rotation, il tend à conserver son orientation dans l’espace, conformément à la loi :\\
\begin{adjustwidth}{6cm}{0cm}$\dfrac{d\vec{L}_O}{dt} = 0 \Rightarrow \vec{L}_O = \vec{\text{Cte}}$
\end{adjustwidth}
\vspace{0.1cm}\\
avec $\vec{L}_O$ le moment cinétique par rapport à un point O.
\vspace{0.5cm}\\
Dans les gyroscopes MEMS (Micro-Electro-Mechanical Systems) modernes, cette rotation est généralement mesurée par effet Coriolis. Une petite structure oscillante à l’intérieur du capteur subit une déviation lorsqu’elle est en mouvement de rotation. Cette déviation est proportionnelle à la vitesse angulaire et peut être détectée électriquement.
\vspace{0.5cm}\\
Grâce à ces mesures, le gyroscope est capable de suivre les changements d’orientation d’un appareil, ce qui le rend essentiel pour notre processus de développement.

\subsection{Magnétomètre}
Le magnétomètre est un capteur qui mesure l’intensité et la direction d’un champ magnétique, généralement celui de la Terre. Il fonctionne comme une boussole numérique, permettant de déterminer l’orientation absolue d’un objet par rapport au nord magnétique. Contrairement au gyroscope, qui mesure des vitesses de rotation relatives, le magnétomètre fournit un repère fixe dans l’espace.
\vspace{0.5cm}\\
Le principe de fonctionnement le plus courant repose sur l’effet Hall. Lorsqu’un courant électrique traverse un conducteur ou un semi-conducteur soumis à un champ magnétique perpendiculaire, une différence de potentiel apparaît sur les côtés du matériau. Cette tension, appelée tension de Hall, est proportionnelle à l’intensité du champ magnétique traversant le capteur. Elle est mesurée électroniquement pour en déduire la composante du champ magnétique selon chaque axe (X, Y, Z).
\vspace{0.5cm}\\
Pour cette raison, il est souvent combiné avec un gyroscope et un accéléromètre, permettant une estimation plus fiable et précise de l’orientation dans l’espace.


\subsection{LiDAR}
Le LiDAR (Light Detection and Ranging) est un capteur qui mesure la distance entre un capteur et un objet en utilisant des impulsions lumineuses, généralement des lasers. Il permet de cartographier avec précision l’environnement en trois dimensions en mesurant le temps mis par la lumière pour revenir après avoir été réfléchie par un objet.
\vspace{0.5cm}\\
Le principe de fonctionnement repose sur la mesure du temps de vol (Time of Flight) d’un faisceau laser. Le capteur émet une impulsion lumineuse, puis détecte le retour de celle-ci après réflexion. En connaissant la vitesse de la lumière, il est possible de calculer la distance avec une grande précision. Cette méthode permet d’obtenir un nuage de points représentant la géométrie de l’environnement.
\vspace{0.5cm}\\
Grâce à sa capacité à fournir des mesures précises, même dans des conditions de faible luminosité, le LiDAR est largement utilisé en robotique, en cartographie et dans les véhicules autonomes pour la détection d’obstacles et la navigation.


\newpage
\section{Avantages et inconvénients}
Chaque capteur mesure une grandeur physique, mais aucun capteur n'est parfait. \\
Chacun présente des atouts et des limitations qu'il convient de connaitre afin de concevoir un système de mesure fiable.
\begin{itemize}
    \item \textbf{GPS} : \begin{itemize}
        \item \textit{Avantages} : \\ Le GPS a une bonne précision géographique (de l'ordre de quelques mètres pour un GPS civil).\\ De plus, il fonctionne de manière autonome. En effet, il n'a pas
        besoin d'infrastructure locale. En outre, il permet de déterminer la position absolue sur le globe (latitude, longitude, altitude).
        \item \textit{Limites} : \\ Même si le GPS nous offre plusieurs avantages. Ce dernier a quelques limites à ne pas ignorer. En fait, il a une latence élevée de l'ordre de plusieurs centaines de millisecondes à quelques secondes.\\
        Le GPS a aussi une faible fréquence d'échantillonnage (entre 1Hz et 10Hz), ce qui n'est parfois pas suffisant surtout pour capter les dynamiques rapides. Il ne faut pas oublier aussi qu'il est peu fiable
        en environnements clos comme les bâtiments, tunnels, etc... car il nécessite une ligne de vue avec les satellites. Finalement, le GPS est vulnérable aux interférences radio et au brouillage.
    \end{itemize}
    \item \textbf{Accéléromètre} \begin{itemize}
        \item \textit{Avantages} : \\ L'accélèromètre nous offre une mesure directe de l'accélération linéaire dans les 3 axes (x,y,z). En effet, ce dernier est très utile pour détecter les chocs, les mouvements  ou les vibrations.
        Sa haute fréquence d'échantillonnage (plusieurs KHz) lui permet d'avoir un suivi précis des changements rapides. Les avantages les plus importants de l'accélèromètre sont sa taille et son coût. En effet, il est peu coûteux et très compact.

        \item \textit{Limites} : \\ Tout comme le GPS, l'accélèromètre a ses inconvénients.En fait, un accélèromètre est très sensible au bruit (bruit thermique, vibrations parasites).
        Les mesures nécessitent souvent un filtrage notamment : un filtre passe-bas ou filtre de Kalman. De plus, un accélèromètre ne donne pas la position absolue. Il donne seulement des informations relatives au mouvement.

    \end{itemize}
    \item \textbf{Gyroscope} : 
    \begin{itemize}
        \item \textit{Avantages} : \\ Le gyroscope permet de mesurer la vitesse angulaire, c’est-à-dire les rotations autour des axes. \\ Il est très réactif et précis à court terme, ce qui le rend particulièrement utile pour détecter des changements rapides d’orientation. \\ Grâce à sa haute fréquence d’échantillonnage, il est capable de suivre les mouvements avec une grande précision en temps réel.
        
        \item \textit{Limites} : \\ Cependant, le gyroscope présente aussi certaines limites. \\ Il souffre notamment d’un phénomène appelé dérive : les petites erreurs de mesure s’accumulent avec le temps, ce qui dégrade la précision à long terme si le capteur n’est pas recalibré. \\ De plus, il consomme généralement plus d’énergie que d’autres capteurs comme l’accéléromètre, ce qui peut poser problème dans les systèmes autonomes. 
    \end{itemize}
    
    \item \textbf{Magnétomètre} : 
    \begin{itemize}
        \item \textit{Avantages} : \\ Le magnétomètre mesure le champ magnétique terrestre, ce qui permet d’estimer l’orientation absolue par rapport au nord magnétique. \\ Contrairement au gyroscope, il ne dérive pas avec le temps. \\ Il est donc particulièrement utile pour corriger les erreurs d’orientation dans les systèmes de navigation. 
        
        \item \textit{Limites} : \\ Le principal inconvénient du magnétomètre est sa grande sensibilité aux perturbations magnétiques locales. \\ En présence de métaux ferromagnétiques ou de champs électromagnétiques générés par des appareils électroniques, ses mesures peuvent devenir très imprécises. \\ Il nécessite souvent un étalonnage régulier pour fonctionner correctement, notamment en environnement intérieur ou urbain dense.
    \end{itemize}

    \item \textbf{LiDAR} :  
\begin{itemize}
    \item \textit{Avantages} : \\ 
    Le LiDAR permet de mesurer avec une grande précision la distance aux objets en utilisant des impulsions lumineuses. \\ 
    Il génère un nuage de points 3D détaillé, utile pour la cartographie, la détection d’obstacles et la modélisation de l’environnement. \\ 
    Contrairement aux caméras, il fonctionne efficacement dans des conditions de faible luminosité ou d’obscurité totale.

    \item \textit{Limites} : \\ 
    Le LiDAR peut être sensible aux conditions atmosphériques (pluie, brouillard, poussière) qui affectent la propagation des impulsions lumineuses. \\ 
    Il peut également être coûteux, énergivore, et sa portée est limitée par la puissance du faisceau et les propriétés réfléchissantes des surfaces. \\ 
    Les surfaces transparentes ou très absorbantes peuvent entraîner des erreurs ou des absences de mesure.
\end{itemize}


\end{itemize}

\newpage
\section{Sources de bruitages et/ou défauts techniques}
Les capteurs sont sujets à différents types de bruits :
\item \textbf{Sources de bruit des capteurs} :
\begin{itemize}
    \item \textit{GPS} : \ Bien que largement utilisé, le GPS reste sensible à divers bruits et erreurs. Les conditions atmosphériques, comme les variations dans la troposphère et l’ionosphère, peuvent déformer les signaux satellites, entraînant des erreurs de position. En zone urbaine ou montagneuse, les obstacles bloquent les signaux ou limitent la visibilité des satellites, réduisant la précision. De plus, le phénomène de multipath, où les signaux rebondissent avant d’atteindre le récepteur, fausse le calcul des distances.

    \item \textit{Accéléromètre et gyroscope} : \
Les capteurs inertiels souffrent d’erreurs internes dues à leur structure physique et électronique. Le \textit{bruit thermique}, causé par l’agitation des électrons, génère des fluctuations aléatoires. Les \textit{biais}, erreurs constantes influencées par la température ou l’usure, s’accumulent lors de l’intégration des mesures, entraînant une \textit{dérive} : le système détecte un mouvement inexistant. S’y ajoutent les erreurs d’alignement et les vibrations mécaniques, rendant ces capteurs peu fiables seuls sur le long terme sans recalibrage ou fusion de données.

    \item \textit{Magnétomètre} : \
Ce capteur mesure le champ magnétique terrestre, utile pour estimer l’orientation absolue (nord-sud). Cependant, sa fiabilité dépend fortement de l’environnement. Les objets \textit{ferromagnétiques} et les sources \textit{électromagnétiques} (moteurs, câbles, etc.) peuvent perturber la mesure. En intérieur, ces interférences sont fréquentes. Un \textit{étalonnage} régulier, par mouvements circulaires, est nécessaire pour corriger ces erreurs. Malgré cela, le magnétomètre reste peu fiable seul pour l’orientation.

\item \textit{LiDAR} : \\

Le LiDAR mesure la distance aux objets en envoyant des impulsions laser et en analysant leur temps de vol. Il fournit une représentation 3D précise de l’environnement. Cependant, plusieurs sources de bruit ou défauts techniques peuvent affecter la qualité des mesures. 

Les conditions atmosphériques telles que le brouillard, la pluie ou la poussière dispersent ou absorbent partiellement le faisceau laser, réduisant la portée effective et la précision du retour. Les surfaces très réfléchissantes (comme le verre ou les miroirs) peuvent provoquer des réflexions multiples, entraînant des mesures erronées ou décalées. À l’inverse, les matériaux sombres ou absorbants (comme les tissus noirs ou certaines peintures mates) peuvent réfléchir très peu de lumière, générant des données incomplètes ou du bruit.

Des erreurs peuvent aussi survenir à cause de la géométrie de la scène : des angles trop inclinés par rapport au capteur réduisent la puissance du signal réfléchi. De plus, les vibrations du support ou des erreurs de synchronisation dans les systèmes multi-capteurs peuvent introduire du flou ou des artefacts dans le nuage de points. Enfin, les LiDARs à balayage mécanique peuvent s’user ou se désaligner avec le temps, affectant leur précision.

Malgré ces limitations, le LiDAR reste l’un des capteurs les plus fiables pour la perception 3D, notamment lorsqu’il est utilisé en complément d’autres capteurs.

\end{itemize}



\section{Calibration effectuée}

\subsection{Démarche}
La calibration vise à réduire les erreurs systématiques. Pour l’accéléromètre et le gyroscope, nous avons mesuré les valeurs à l’arrêt pour corriger les biais. 

\subsection{Expérimentations}
Nous avons relevé les valeurs brutes dans différentes orientations et positions. Les écarts ont permis d’ajuster les offsets et les gains.

\subsection{Résultats}
Les données corrigées présentent une réduction notable du biais et une meilleure cohérence des mesures, notamment lors des déplacements rectilignes ou rotations contrôlées.

% Choix des paramètres d'acquisition
\newpage
\section{Choix des paramètres d'acquisition}

Nous avons adopté une démarche statistique pour le choix des paramètres d'acquisition. Vingt séries de mesures ont été réalisées en faisant varier au maximum les réglages des différents capteurs. Néanmoins, l'absence de référence externe (le smartphone) pour comparer nos résultats a posé la question suivante : comment choisir un bon paramétrage sans pouvoir évaluer précisément les erreurs systématiques ?

\subsection{Test d'homogénéité des variances}

Dans un premier temps, nous avons évalué l’impact des modifications de paramètres sur la dispersion des mesures à l’aide du test de Levene (https://datatab.fr/tutorial/levene-test). Ce test permet de vérifier l’hypothèse nulle selon laquelle plusieurs échantillons proviennent d’une population à variance identique. De plus, il accepte des échantillons de tailles différentes.

\begin{itemize}
  \item $H_0$~: les groupes ont des variances égales.
  \item $H_1$~: les groupes ont des variances différentes.
\end{itemize}

Si la p-valeur obtenue est supérieure à 0,05, on ne peut pas rejeter $H_0$ : les variances sont jugées homogènes. Si elle est inférieure à 0,05, on conclut à une différence significative des variances.

Les hypothèses du test de Levene sont :
\begin{itemize}
  \item observations indépendantes,
  \item variable mesurée à un niveau d'échelle métrique.
\end{itemize}

Nous avons implémenté ce test en Python, puis comparé nos résultats à ceux fournis par la fonction \texttt{levene} de la bibliothèque \texttt{SciPy} (cf. annexe).

\subsection{Sélection empirique du meilleur paramétrage}

Une fois l’impact des réglages mis en évidence, nous avons adopté une approche empirique : parmi les vingt acquisitions, nous avons sélectionné celle présentant l’écart-type le plus faible pour chaque capteur, indiquant la meilleure stabilité des mesures.

Cette méthode, bien que simple, permet de retenir un paramétrage optimisé en termes de précision, tout en restant compatible avec les contraintes énergétiques et informatiques évoquées précédemment.


\section{Filtrage des signaux bruts}

Le filtrage des données brutes consiste à effectuer un prétraitement afin d'améliorer la qualité des signaux issus des capteurs. L'objectif principal est de garantir la cohérence des acquisitions et de supprimer les valeurs aberrantes (outliers) susceptibles de fausser les analyses ou traitements ultérieurs.

\subsection{Synchronisation des signaux}

Avant tout traitement, nous avons procédé à la synchronisation temporelle des différentes acquisitions issues des capteurs (gyroscope, accéléromètre, magnétomètre, etc.). Cela permet de garantir que les mesures analysées correspondent aux mêmes instants physiques, ce qui est essentiel pour toute fusion ou comparaison de données multisources. La synchronisation a été réalisée en alignant les horodatages fournis par chaque capteur.

\subsection{Détection des valeurs aberrantes}

Une fois les données synchronisées, nous avons utilisé des représentations graphiques de type \textbf{boxplot} (ou boîtes à moustaches) pour visualiser la distribution des données de chaque capteur. Ces graphiques permettent d’identifier facilement les éventuelles valeurs aberrantes, c’est-à-dire les points très éloignés de la médiane ou dépassant l’intervalle interquartile étendu (défini en cours par $[Q1 - 1{,}5 \times IQR,\ Q3 + 1{,}5 \times IQR]$).

Ces valeurs peuvent provenir de bruits ponctuels, de pertes de signal ou d'erreurs transitoires dans la transmission. Elles ont été exclues des jeux de données à l'aide d’un filtre conditionnel simple basé sur ces bornes statistiques.

\subsection{Lissage et filtrage complémentaire}

En complément, un filtrage numérique a pu être appliqué sur certaines séries temporelles à fort bruit, notamment via un filtre passe-bas de type moyenne glissante (moving average) ou un filtre de Savitzky-Golay, selon le type de signal et le niveau de bruit observé. Cela permet de réduire les fluctuations rapides non représentatives sans altérer les tendances générales du mouvement.

\subsection{Résultat du prétraitement}

Après filtrage, les signaux présentent une meilleure continuité et une diminution significative du bruit. Les valeurs extrêmes ont été supprimées, rendant les signaux plus exploitables pour les étapes suivantes telles que la calibration dynamique ou la fusion multi-capteurs.


\chapter{Présentation du système d’acquisition complet}

\section{Montage électrique}

Le montage électrique mis en place repose sur l’intégration d’une Raspberry Pi et de capteurs connectés, associés à plusieurs équipements liés à la collecte des données. Les principaux composants connectés au système sont :

\begin{itemize}
    \item un module GPS,
    \item un accéléromètre,
    \item un écran LCD
    \item Lidar
\end{itemize}

L’assemblage nécessite une attention particulière au niveau des branchements afin d’éviter toute perte de potentiel ou de court-circuit. Un schéma de câblage est également présenté ci-dessous:

SHCEMAAAAAAAAAAAAAAAAAAAAAAAAAAAAAAAAAAAAAAAAAAAAAAAAAAAAAAAAAAAAAAAAAAAAAAAAAAAAAAAAAAAAAAAAAAAAAAAAAAAAAAAAAAAAAA

\section{Description du produit (fonctionnalités, mode d’utilisation)}

L’application développée atteint pleinement l’objectif initial du projet : proposer un système permettant de suivre son trajet sur une carte de manière interactive, à la manière de Strava.\\
Elle intègre une fonction de télémétrie pour comparer différents parcours (temps moyen, distance, vitesse, etc.) et permet un suivi précis grâce à l’intégration en temps réel d’un filtre de Kalman, combinant les données GPS et IMU pour améliorer la précision du positionnement.

En complément de ces fonctionnalités, une option de course d’orientation a été ajoutée. Elle permet à l’utilisateur de suivre un itinéraire défini par des points de passage, avec un système de validation à chaque balise franchie, offrant ainsi un usage ludique et sportif du dispositif.

\section{Synchronisation des capteurs}

La synchronisation des capteurs a été assurée à l’aide de l’heure UTC comme référence temporelle commune. Chaque échantillon de données, qu’il provienne du GPS ou de l’IMU, est horodaté avec un \textit{timestamp} précis. Cela permet d’aligner chronologiquement les mesures issues des différents capteurs.

Grâce à cette méthode, nous avons pu constituer un jeu de données unifié, combinant de manière cohérente les informations GPS et IMU pour un traitement fiable, notamment lors de l'application du filtre de Kalman.

\section{Fusion de données détaillée}

On a principalement utilisé le filtre de Kalman (Voir Modélisation et Estimation par Filtre de Kalman du Chapitre 4).


\chapter{Analyse des performances du système d'acquisition}

\section{Méthodologie pour la détermination du paramétrage du système d'acquisition}

Pour déterminer les meilleurs réglages de nos capteurs, nous avons réalisé plusieurs acquisitions du même trajet, en changeant les paramètres à chaque fois. Cela nous a permis de comparer les résultats et d’évaluer l’impact de chaque réglage sur la qualité des données.

Les paramètres modifiés sont les suivants :

\begin{itemize}
  \item \textbf{Accéléromètre}
  \begin{itemize}
    \item Plages : $\pm2g$, $\pm4g$, $\pm8g$, $\pm16g$
    \item Fréquences : 12.5 Hz, 208 Hz, 833 Hz, 1660 Hz, 3330 Hz
  \end{itemize}

  \item \textbf{Gyroscope}
  \begin{itemize}
    \item Plages : $\pm250$ dps, $\pm500$ dps, $\pm1000$ dps, $\pm2000$ dps
    \item Fréquences : 12.5 Hz, 208 Hz, 833 Hz, 1660 Hz, 3330 Hz
  \end{itemize}

  \item \textbf{Magnétomètre}
  \begin{itemize}
    \item Plages : $\pm4$ gauss, $\pm8$ gauss, $\pm12$ gauss, $\pm16$ gauss
    \item Fréquences : 0.625 Hz, 5 Hz, 20 Hz, 150 Hz, 560 Hz
  \end{itemize}
\end{itemize}

Toutes les acquisitions ont été faites sur le même parcours, mais à des jours différents. Les conditions météorologiques n’étaient donc pas les mêmes à chaque fois. Il est donc important de prendre ceci en compte, car cela peut influencer les mesures en introduisant une source de bruit supplémentaire.

Avant de commencer une acquisition, nous avons toujours laissé la carte d’acquisition immobile pendant quelques secondes pour que les capteurs puissent se stabiliser. Cependant, cela a été fait à proximité d’un bâtiment, ce qui n’est pas idéal, en particulier pour le magnétomètre qui peut être perturbé par les structures métalliques aux alentours. Un axe d’amélioration serait donc de débuter les acquisitions dans un environnement dégagé, loin des bâtiments, pour limiter ces perturbations.

\section{Description et analyse du choix des méthodes statistiques avec les résultats obtenus}

Nous avons implémenté manuellement le test de Levene afin de vérifier l’homogénéité des variances entre plusieurs acquisitions. Pour valider notre approche, nous avons comparé les résultats obtenus à ceux fournis par la fonction \texttt{levene} de la bibliothèque \texttt{scipy}. Voici les différentes étapes du processus.

\vspace{0.5cm}

\textbf{1. Importation des acquisitions} \\
Nous avons commencé par importer les différentes séries de mesures à comparer :

\begin{center}
\includegraphics[width=1\linewidth]{sections/Annexes/images_annexes/code_stats_1.png}
\end{center}

\newpage

\textbf{2. Implémentation manuelle du test} \\
Nous avons ensuite codé le test de Levene étape par étape, en calculant les écarts aux médianes puis les sommes de carrés nécessaires à l’évaluation statistique :

\begin{center}
\includegraphics[width=1\linewidth]{sections/Annexes/images_annexes/code_stats_2.png}
\end{center}

\newpage

\textbf{3. Validation avec \texttt{scipy}} \\
Pour s’assurer de la validité de notre implémentation, nous avons comparé nos résultats à ceux de \texttt{scipy.stats.levene}. Une fois cette étape confirmée, nous avons utilisé les résultats pour sélectionner les meilleurs réglages :

\begin{center}
\includegraphics[width=1\linewidth]{sections/Annexes/images_annexes/code_stats_3.png}
\end{center}

\vspace{0.5cm}

\noindent
À l’issue de cette analyse, nous avons retenu les acquisitions suivantes :
\begin{itemize}
    \item Acquisition 13 pour l’accéléromètre,
    \item Acquisition 6 pour le gyroscope,
    \item Acquisition 12 pour le magnétomètre.
\end{itemize}

\section{Analyse des résultats de tracking obtenus}

\subsection{Mouvement extérieur 1}

\subsubsection{Comparaison des trajets}
\begin{figure}[H]
    \centering
    \label{fig: Comparaison trajet mouvement exterieur 1}
    \includegraphics[height=400pt, width = 500pt]{../traitements/mouvement_exterieur1/comparaison_trajets}
    \caption{Comparaison trajet mouvement extérieur 1 (Gps vs Téléphone)}
\end{figure}

Dans cette exemple, on a fait deux tours à l'extérieur et on remarque qu'en prenant les données du téléphone comme repère. Le GPS qu'on utilise n'est pas
trop mauvais. En effet, quand on fait la moyenne des deux trajectoires on retrouve une trajectoire qui ressemble au chemin pris lors de l'acquisition.

\subsubsection{Comparaison latitude  et longitude}
\begin{figure}[h]
    \centering
    \begin{minipage}{0.45\textwidth}
        \centering
        \includegraphics[height = 250pt, width=\linewidth]{../traitements/mouvement_exterieur1/comparaison_latitude}
        \caption{ Comparaison latitude (Gps vs Téléphone) }
    \end{minipage}
    \hfill
    \begin{minipage}{0.45\textwidth}
        \centering
        \includegraphics[height = 250pt , width=\linewidth]{../traitements/mouvement_exterieur1/comparaison_longitude}
        \caption{ Comparaison longitude (Gps vs Téléphone) mouvement exterieur1}
    \end{minipage}
\end{figure}

On remarque que les mesures de latitude et de longitude effectuées à l'aide de notre GPS correspondent pluttôt bien aux mesures réalisées au téléphone sur l'application GPS Logger.\\
On constate cependant qu'en général les mesures prises par l'application mobile présentent des valeurs maximales ( respectivement minimales ) légèrement supérieures ( respectivement inférieures ).

\subsubsection{Boite à moustache vitesse / accéleration}

\begin{figure}[H]
    \centering
    \includegraphics[height=250pt, width = \linewidth]{../traitements/mouvement_exterieur1/boxplots_vitesse_acceleration}
    \caption{Boite à moustache }
    \label{fig: Boxplot mouvement_exterieur1}
\end{figure}\\
On observe une vitesse de marche moyenne de 1 mètre par seconde qui correspond à une vitesse moyenne de 4 kilomètre par heure environ sachant que la seconde boite à moustache présente une moyenne d'accéleration nulle ce qui signifie que l'ensemble des acquisitions ont été effectuées dans les mêmes conditions exactement.



\subsubsection{Evolution de l'erreur mouvement exterieur1}

\begin{figure}[H]
    \centering
    \includegraphics[height=300pt, width = 400pt]{../traitements/mouvement_exterieur1/erreur_evolution}
    \caption{ Evolution de l'erreur en cours du temps }
    \label{fig: erreur_evolution  mouvement_exterieur1}
\end{figure}
Ce graphe illustre l’évolution de l’erreur de position entre les mesures GPS et la position estimée par le téléphone durant un mouvement extérieur. On observe que l’erreur initiale est relativement élevée, atteignant environ 20 mètres, avant de diminuer de manière significative. Par la suite, l’erreur varie de façon irrégulière mais reste généralement comprise entre 2 et 15 mètres. Ces fluctuations peuvent être dues à plusieurs facteurs, tels que des interférences satellites, des obstacles environnementaux (bâtiments, arbres), ou encore des limitations du capteur du téléphone.

Globalement, on remarque une tendance à la stabilisation partielle de l’erreur après les premières observations, ce qui pourrait indiquer un temps d’adaptation ou de calibration du système GPS du téléphone. Ainsi, il faut attendre quelques secondes pour stabiliser avant de commencer nos prochaines acquisitions.


\subsubsection{Valeurs aberrantes}

\begin{figure}[H]
    \centering
    \includegraphics[height=250pt, width = \linewidth]{../traitements/mouvement_exterieur1/valeurs_aberrantes}
    \caption{ Valeurs aberrantes}
    \label{fig: Valeurs aberrantes mouvement_exterieur1}
\end{figure}
La figure~\ref{fig: Valeurs aberrantes mouvement_exterieur1} présente deux graphiques illustrant la détection de \textbf{valeurs aberrantes} dans les séries temporelles de \textbf{vitesse} (graphique intitulé \og Vitesse avec valeurs aberrantes \fg) et d'\textbf{accélération} (graphique intitulé \og Accélération avec valeurs aberrantes \fg) mesurées lors de notre acquisition.

Le graphique \og Vitesse avec valeurs aberrantes \fg montre l’évolution de la vitesse au cours du temps. La courbe bleue représente les vitesses considérées comme normales, tandis que les points rouges indiquent les \textit{valeurs aberrantes} détectées. Ces dernières apparaissent de manière sporadique, souvent aux extrémités de la plage des données. Cela peut refléter des erreurs de capteurs ou des bruits de mesure. La majorité des valeurs se concentrent autour de 1~m/s, ce qui est cohérent avec une vitesse de marche humaine (cf Figure \ref{fig: Boxplot mouvement_exterieur1}).

Le graphique \og Accélération avec valeurs aberrantes \fg montre quant à lui l’évolution de l’accélération. La courbe verte indique les valeurs normales, tandis que les points rouges représentent les accélérations considérées comme aberrantes. On y observe une plus grande densité d'anomalies, ce qui est attendu puisque l’accélération, étant une dérivée seconde, est plus sensible au bruit. Les valeurs normales d’accélération sont généralement proches de zéro, tandis que les pics extrêmes sont identifiés comme des anomalies.

Ainsi, cette figure met en évidence l’importance de la détection de valeurs aberrantes pour améliorer la qualité et la fiabilité des mesures avant tout traitement ou analyse avancée.



\section{Analyse des résultats obtenus en cas de défaillance capteur}
\subsection{Acquisition immobile exterieur 1}
\subsubsection{Comparaison trajet Gps vs Téléphone}
\begin{figure}[H]
    \centering
    \includegraphics[height=400pt, width = 500pt]{../traitements/immobile_exterieur1/comparaison_trajets}
    \caption{Comparaison trajet immobile exterieur 1 (Gps vs Téléphone)}
    \label{fig: Comparaison trajet immobile exterieur 1}
\end{figure}
On remarque que même à l'extérieur, les acquisitions du gps et du téléphone ne sont pas assez précises. Le trajet nous indique qu'on était en mouvement même
si on était immobile.

\subsubsection{Comparaison latitude  et longitude}
\begin{figure}[h]
    \centering
    \begin{minipage}{0.45\textwidth}
        \centering
        \includegraphics[height = 250pt, width=\linewidth]{../traitements/immobile_exterieur1/comparaison_latitude}
        \caption{ Comparaison latitude (Gps vs Téléphone) }
    \end{minipage}
    \hfill
    \begin{minipage}{0.45\textwidth}
        \centering
        \includegraphics[height = 250pt , width=\linewidth]{../traitements/immobile_exterieur1/comparaison_longitude}
        \caption{ Comparaison longitude (Gps vs Téléphone)}
    \end{minipage}
\end{figure}

\subsubsection{Boite à moustache vitesse / accéleration}

\begin{figure}[H]
    \centering
    \includegraphics[height=250pt, width = \linewidth]{../traitements/immobile_exterieur1/boxplots_vitesse_acceleration}
    \caption{Boite à moustache }
    \label{fig: Boxplot immobile exterieur 1}
\end{figure}


\subsubsection{Evolution de l'erreur }

\begin{figure}[H]
    \centering
    \includegraphics[height=300pt, width = 400pt]{../traitements/immobile_exterieur1/erreur_evolution}
    \caption{ Evolution de l'erreur en cours du temps }
    \label{fig: erreur_evolution immobile exterieur }
\end{figure}


\subsubsection{Valeurs aberrantes}

\begin{figure}[H]
    \centering
    \includegraphics[height=250pt, width = \linewidth]{../traitements/immobile_exterieur1/valeurs_aberrantes}
    \caption{ Valeurs aberrantes }
    \label{fig: Valeurs aberrantes immobile exterieur}
\end{figure}

\subsection{ Acquisition immobile exterieur 2 }

\begin{figure}[H]
    \centering
    \includegraphics[height=400pt, width = 500pt]{../traitements/immobile_exterieur2/comparaison_trajets}
    \caption{Comparaison trajet immobile exterieur 2 (Gps vs Téléphone)}
    \label{fig: Comparaison trajet immobile exterieur 2}
\end{figure}

Ici, on peut remarquer que les données fournies par le téléphone sont beaucoup plus précises que les données fournies par le GPS. En effet, on ne peut pas se fier
aux données du GPS. (à conclure après je sais pas voir avec la démarche etc)


\subsubsection{Comparaison latitude  et longitude}
\begin{figure}[h]
    \centering
    \begin{minipage}{0.45\textwidth}
        \centering
        \includegraphics[height = 250pt, width=\linewidth]{../traitements/immobile_exterieur2/comparaison_latitude}
        \caption{ Comparaison latitude (Gps vs Téléphone) immobile exterieur 2}
    \end{minipage}
    \hfill
    \begin{minipage}{0.45\textwidth}
        \centering
        \includegraphics[height = 250pt , width=\linewidth]{../traitements/immobile_exterieur2/comparaison_longitude}
        \caption{ Comparaison longitude (Gps vs Téléphone) immobile exterieur 2 }
    \end{minipage}
\end{figure}

\subsubsection{Boite à moustache vitesse / accéleration}

\begin{figure}[H]
    \centering
    \includegraphics[height=250pt, width = \linewidth]{../traitements/immobile_exterieur2/boxplots_vitesse_acceleration}
    \caption{Boite à moustache }
    \label{fig: Boxplot immobile exterieur 2}
\end{figure}


\subsubsection{Evolution de l'erreur }

\begin{figure}[H]
    \centering
    \includegraphics[height=300pt, width = 400pt]{../traitements/immobile_exterieur2/erreur_evolution}
    \caption{ Evolution de l'erreur en cours du temps }
    \label{fig: erreur_evolution immobile exterieur 2 }
\end{figure}


\subsubsection{Valeurs aberrantes}

\begin{figure}[H]
    \centering
    \includegraphics[height=250pt, width = \linewidth]{../traitements/immobile_exterieur2/valeurs_aberrantes}
    \caption{ Valeurs aberrantes }
    \label{fig: Valeurs aberrantes immobile exterieur 2}
\end{figure}

\documentclass[12pt,a4paper]{article}

\usepackage[utf8]{inputenc}
\usepackage[T1]{fontenc}
\usepackage[french]{babel}
\usepackage{amsmath, amssymb}
\usepackage{geometry}
\usepackage{graphicx}
\usepackage{hyperref}

\geometry{margin=2.5cm}

\title{Modélisation et Estimation par Filtre de Kalman}
\author{}
\date{}

\begin{document}

\maketitle
\section*{Données disponibles}
Dans le cadre de ce projet, les données accessibles sont les suivantes :
\begin{itemize}
    \item \textbf{Position GPS :}
    \begin{itemize}
        \item $\theta$ : longitude (coordonnée est-ouest)
        \item $\lambda$ : latitude (coordonnée nord-sud)
    \end{itemize}

    \item \textbf{Données de l'IMU (Inertial Measurement Unit) :}
    \begin{itemize}
        \item \textbf{Accéléromètre :} Mesures d'accélération linéaire
        \begin{itemize}
            \item $a_x$ : accélération selon l'axe $x$
            \item $a_y$ : accélération selon l'axe $y$
            \item $a_z$ : accélération selon l'axe $z$
        \end{itemize}

        \item \textbf{Gyroscope :} Mesures de vitesse angulaire
        \begin{itemize}
            \item $\dot{\omega}_x$ : vitesse angulaire autour de l'axe $x$ (roulis)
            \item $\dot{\omega}_y$ : vitesse angulaire autour de l'axe $y$ (tangage)
            \item $\dot{\omega}_z$ : vitesse angulaire autour de l'axe $z$ (lacet)
        \end{itemize}

        \item \textbf{Magnétomètre :} Mesures du champ magnétique terrestre
        \begin{itemize}
            \item $m_x$ : composante selon l'axe $x$
            \item $m_y$ : composante selon l'axe $y$
            \item $m_z$ : composante selon l'axe $z$
        \end{itemize}
    \end{itemize}
\end{itemize}

\subsection*{Remarques}
\begin{itemize}
    \item Les données GPS ($\theta$, $\lambda$) donnent la position absolue, mais avec une précision limitée (erreur de quelques mètres).
    \item Les données de l'IMU fournissent des mesures inertielles précises à court terme, mais sujettes à une dérive temporelle (erreur qui s'accumule). De plus, l'accélération s'exprime dans le repère de l'IMU lui-même et non dans un repère terrestre.
    \item Le magnétomètre permet de s'orienter par rapport au nord magnétique, mais peut être perturbé par des interférences locales.
\end{itemize}

\section*{Premier modèle basique}
En raison de la nature instable de certains de nos composants, nous serions tentés de créer des modèles de filtre de Kalman compliqués afin de pouvoir corriger les différentes erreurs qui s'accumulent. Mais pour débuter, nous souhaitons faire un modèle simple, linéaire, qui ne nécessite pas l'utilisation d'un filtre de Kalman étendu.

\subsection*{Rappel filtre de Kalman}
En statistique et en théorie du contrôle, le filtre de Kalman est un estimateur récursif à réponse impulsionnelle infinie permettant d'estimer les états d'un système dynamique à partir de mesures bruitées ou incomplètes \cite{wikip_kalman}.

Le filtre de Kalman en contexte discret est un estimateur récursif : l'état courant est estimé à partir de l'estimation de l'état précédent et des mesures actuelles. Le filtre de Kalman suppose que le processus discret réel $\mathbf{x}_{k}$ (où $k$ dénote l'indice de temps), suit la loi d'évolution linéaire suivante :
\[
\mathbf{x}_{k} = \mathbf{F}_{k}\mathbf{x}_{k-1} + \mathbf{G}_{k}\mathbf{u}_{k} + \mathbf{w}_{k}
\]
où :
\begin{itemize}
    \item $\mathbf{F}_k$ : matrice de transition entre l'état $k-1$ et l'état $k$
    \item $\mathbf{u}_k$ : commande d'entrée
    \item $\mathbf{G}_k$ : matrice de contrôle reliant $\mathbf{u}_k$ et $\mathbf{x}_k$
    \item $\mathbf{w}_{k}$ : bruit d'évolution, gaussien centré de covariance $\mathbf{Q}_{k}$
\end{itemize}

L'observation $\mathbf{z}_k$ à l'instant $k$ est donnée par :
\[
\mathbf{z}_k = \mathbf{H}_k\mathbf{x}_k + \mathbf{v}_k
\]
avec :
\begin{itemize}
    \item $\mathbf{H}_k$ : matrice d'observation à l'instant $k$
    \item $\mathbf{v}_k$ : bruit de mesure
\end{itemize}

\subsection*{Prétraitement des données}
Maintenant que nous savons quels éléments nous devons définir ainsi que les données que nous avons à notre disposition, nous pouvons modéliser le système.
Cependant, un problème apparaît : nos acquisitions sont en coordonnées sphériques. Cela complique les traitements, qui risquent de ne pas être linéaires. Il est donc préférable de les transformer en coordonnées cartésiennes pour pouvoir appliquer un modèle linéaire, notamment pour l'intégration dans un filtre de Kalman classique.

Pour convertir les coordonnées sphériques en coordonnées cartésiennes, on utilise :
\begin{align*}
x &= R_t \cos(\lambda) \cos(\theta) \\
y &= R_t \cos(\lambda) \sin(\theta) \\
z &= R_t \sin(\lambda)
\end{align*}
avec :
\begin{itemize}
    \item $R_t$ : rayon terrestre
    \item $\lambda$ : latitude en radians
    \item $\theta$ : longitude en radians
\end{itemize}

Comme dit plus haut, les accélérations sont exprimées dans le repère local de l'accéléromètre. Pour que notre filtre fonctionne correctement, il faut les exprimer dans le repère terrestre. On applique donc une matrice de rotation à nos vecteurs d'accélération.

Les angles d'orientation sont obtenus par intégration des vitesses angulaires :
\begin{align}
    \omega_{x,k+1} &= \omega_{x,k} + \dot{\omega}_{x,k} \, \Delta t \\
    \omega_{y,k+1} &= \omega_{y,k} + \dot{\omega}_{y,k} \, \Delta t \\
    \omega_{z,k+1} &= \omega_{z,k} + \dot{\omega}_{z,k} \, \Delta t
\end{align}

Conditions initiales :
\[
\omega_{x,0}, \quad \omega_{y,0}, \quad \omega_{z,0} \quad \text{calculés à } t = 0 \text{ avec les données du magnétomètre}
\]

On en déduit la matrice de rotation $R_k$ à appliquer à nos vecteurs d'accélération à l'instant $k$ :
\[
R_k = R_{\omega_{x,k}} R_{\omega_{y,k}} R_{\omega_{z,k}}
\]

Ainsi, on obtient l'accélération dans le repère terrestre :
\[
\tilde{\mathbf{a}} =
\begin{pmatrix}
    \tilde{a}_x \\
    \tilde{a}_y \\
    \tilde{a}_z
\end{pmatrix}
=
R_k
\begin{pmatrix}
    a_x \\
    a_y \\
    a_z
\end{pmatrix}
\]

\subsection*{Modélisation du problème}
Toutes nos données sont désormais transformées pour être utilisables.

On définit le vecteur d'état $\mathbf{x}_k$ :
\[
\mathbf{x}_k =
\begin{pmatrix}
    x \\
    y \\
    z \\
    v_x \\
    v_y \\
    v_z
\end{pmatrix}
\]

À partir des équations classiques du mouvement, on a :
\[
\begin{cases}
x_{k+1} = x_k + v_{x,k} \, dt + \frac{1}{2} \tilde{a}_{x,k} \, dt^2 \\
y_{k+1} = y_k + v_{y,k} \, dt + \frac{1}{2} \tilde{a}_{y,k} \, dt^2 \\
z_{k+1} = z_k + v_{z,k} \, dt + \frac{1}{2} \tilde{a}_{z,k} \, dt^2 \\
v_{x,k+1} = v_{x,k} + \tilde{a}_{x,k} \, dt \\
v_{y,k+1} = v_{y,k} + \tilde{a}_{y,k} \, dt \\
v_{z,k+1} = v_{z,k} + \tilde{a}_{z,k} \, dt
\end{cases}
\]

Ce système est mis sous forme matricielle :
\[
\mathbf{x}_{k+1} = \mathbf{F}_k \mathbf{x}_k + \mathbf{G}_k \mathbf{u}_k + \mathbf{w}_k
\]
avec :
\[
\mathbf{F}_k =
\begin{pmatrix}
    1 & 0 & 0 & dt & 0 & 0 \\
    0 & 1 & 0 & 0 & dt & 0 \\
    0 & 0 & 1 & 0 & 0 & dt \\
    0 & 0 & 0 & 1 & 0 & 0 \\
    0 & 0 & 0 & 0 & 1 & 0 \\
    0 & 0 & 0 & 0 & 0 & 1
\end{pmatrix}
\quad \text{et} \quad
\mathbf{G}_k =
\begin{pmatrix}
    \frac{1}{2} dt^2 & 0 & 0 \\
    0 & \frac{1}{2} dt^2 & 0 \\
    0 & 0 & \frac{1}{2} dt^2 \\
    dt & 0 & 0 \\
    0 & dt & 0 \\
    0 & 0 & dt
\end{pmatrix}
\]
et $\mathbf{u}_k = \tilde{\mathbf{a}}_k$
\\
Enfin, comme on souhaite observer uniquement la position, la matrice d'observation $\mathbf{H}_k$ est :
\[
\mathbf{H}_k =
\begin{pmatrix}
    1 & 0 & 0 & 0 & 0 & 0 \\
    0 & 1 & 0 & 0 & 0 & 0 \\
    0 & 0 & 1 & 0 & 0 & 0
\end{pmatrix}
\]

\section*{Analyse des résultats}

Après implémentation et exécution du filtre de Kalman sur les données disponibles, nous pouvons analyser les résultats obtenus afin d’évaluer la performance et la pertinence de notre modèle basique.

\subsection*{Validation de l’estimation de la position}

Le filtre de Kalman permet d’obtenir une estimation optimisée de la position en combinant la mesure GPS, précise à long terme mais bruitée, avec les mesures inertielle issues de l’IMU, précises à court terme mais sujettes à dérive.

\begin{itemize}
    \item \textbf{Précision spatiale :}
    La comparaison des positions estimées avec les mesures GPS brutes montre généralement une réduction du bruit aléatoire, traduisant un meilleur lissage des trajectoires.
    \item \textbf{Réactivité :}
    Grâce à l’intégration des accélérations corrigées, le filtre peut suivre les variations rapides de mouvement même en l’absence temporaire de signaux GPS (ex. dans des tunnels ou zones d’interférences).
    \item \textbf{Erreur d’estimation :}
    L’erreur quadratique moyenne (RMSE) entre la position estimée et les données GPS de référence peut être calculée pour quantifier l’amélioration obtenue par rapport aux mesures brutes.
\end{itemize}

\subsection*{Limites du modèle basique}

Malgré ces améliorations, certaines limitations inhérentes au modèle simplifié apparaissent :

\begin{itemize}
    \item \textbf{Approximation linéaire :}
    Le modèle linéaire ne prend pas en compte les non-linéarités dues à la dynamique réelle du véhicule ou des capteurs, ce qui peut dégrader les performances en cas de mouvements complexes.
    \item \textbf{Modélisation des bruits :}
    Les matrices de covariance $\mathbf{Q}_k$ (processus) et $\mathbf{R}_k$ (mesure) doivent être choisies avec soin. Un mauvais calibrage peut engendrer une mauvaise estimation, oscillante ou trop lente à réagir.
    \item \textbf{Dérive gyroscopique et erreurs d’orientation :}
    L’intégration des vitesses angulaires pour calculer la matrice de rotation $R_k$ accumule les erreurs au fil du temps, impactant la qualité de la transformation des accélérations et donc l’estimation des vitesses et positions.
    \item \textbf{Perturbations du magnétomètre :}
    Les interférences magnétiques locales peuvent fausser l’orientation initiale et rendre instable la transformation vers le repère terrestre.
\end{itemize}

\subsection*{Perspectives d’amélioration}

Pour pallier ces limites, plusieurs axes d’amélioration peuvent être envisagés :

\begin{itemize}
    \item Utilisation d’un \textbf{filtre de Kalman étendu (EKF)} ou \textbf{filtre de Kalman unscented (UKF)} afin de mieux gérer les non-linéarités dans la dynamique et l’observation.
    \item Intégration d’un modèle plus complet de la dynamique du système, par exemple en incluant les angles d’orientation comme variables d’état, pour mieux estimer la matrice de rotation.
    \item Calibration plus fine des matrices de bruit et étude des modèles statistiques des erreurs spécifiques aux capteurs.
    \item Fusion avec d’autres capteurs, comme un baromètre pour la hauteur, ou des systèmes de correction GPS différentiel pour réduire les erreurs absolues.
\end{itemize}

\subsection*{Conclusion}

Ce premier modèle linéaire simple de filtre de Kalman permet d’établir une base robuste pour la fusion des données GPS et IMU. Il montre une amélioration sensible de la qualité d’estimation de la position par rapport aux mesures brutes. Cependant, les limites relevées invitent à développer un modèle plus complet et adapté aux spécificités des capteurs et de la dynamique étudiée pour une application finale plus précise et fiable.



\bibliographystyle{plain}
\bibliography{refs}

\end{document}

\includepdf[pages={1-7}]{sections/Partie3/kalman.pdf}
\chapter*{Conclusion}

Ce projet nous a permis de concevoir et de réaliser un système complet d’estimation de trajectoire basé sur la fusion de données capteurs, avec pour objectif de proposer une alternative simplifiée à des applications comme Strava.

Nous avons étudié, calibré et synchronisé différents capteurs (GPS, accéléromètre, gyroscope, magnétomètre), tout en développant une application capable d’en exploiter les mesures en temps réel. L’intégration d’un filtre de Kalman a constitué une étape centrale, permettant de combiner efficacement les données GPS et IMU afin d’améliorer la précision de localisation.

Nous avons également enrichi notre système d’une fonctionnalité de course d’orientation, rendant l’usage plus ludique et interactif. Sur le plan méthodologique, nous avons mis en œuvre des outils statistiques rigoureux pour optimiser les paramètres d’acquisition, détecter les valeurs aberrantes et évaluer les performances du système.

Malgré quelques limitations dans la validation expérimentale, dues à un manque de données de référence sur certaines acquisitions, les résultats obtenus sont prometteurs. Ce travail ouvre la voie à des perspectives d’amélioration, notamment en explorant des variantes non linéaires du filtre de Kalman, en affinant les modèles dynamiques ou en intégrant d’autres sources de données.

Enfin, ce projet nous a permis de mobiliser un large éventail de compétences — théoriques, pratiques et collaboratives — tout en nous confrontant à une problématique proche d’un contexte réel d’ingénierie. Il constitue une expérience formatrice riche, tant sur le plan technique que dans la gestion de projet en équipe.




\begin{thebibliography}{9}
    \bibitem{einstein1905}
    Albert Einstein.
    \textit{Zur Elektrodynamik bewegter Körper}.
    Annalen der Physik, 1905.

    \bibitem{kalmanPDF}
    Stéphane Calderon, Floris Chabert.
    \textit{Filtrage de Kalman appliqué à une centrale inertielle multi-capteurs},
    École Nationale Supérieure des Télécommunications, Paris.
    \textit{Disponible à:} \url{https://igns.wdfiles.com/local--files/filtre-de-kalman/Kalman.pdf},
    (Consulté le 01/04/2025).

    \bibitem{wikip_jacobienne}
    \textit{Matrice jacobienne},
    Wikipédia, l'encyclopédie libre.
    \textit{Disponible à:} \url{https://fr.wikipedia.org/wiki/Matrice_jacobienne},
    (Consulté le 01/04/2025).

    \bibitem{wikip_kalman}
    \textit{Filtre de Kalman},
    Wikipédia, l'encyclopédie libre.
    \textit{Disponible à:} \url{https://fr.wikipedia.org/wiki/Filtre_de_Kalman},
    (Consulté le 01/04/2025).
\end{thebibliography}



\end{document}
