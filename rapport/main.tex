\documentclass[12pt,a4paper,openright]{report}

% Encodage et langue
\usepackage[utf8]{inputenc} % Encodage UTF-8
\usepackage[T1]{fontenc} % Encodage des fontes
\usepackage[french,english]{babel} % Langues : français et anglais

% Police et espacement
\usepackage[sc]{mathpazo} % Police Palatino
\linespread{1.05} % Espacement des lignes

% Couleurs et graphiques
\usepackage{xcolor} % Gestion des couleurs
\definecolor{customblue}{RGB}{33,26,82} % Définition de la couleur aaublue
\usepackage{graphicx} % Inclusion d'images
\usepackage{wrapfig} % Figures et tableaux enroulés dans le texte
\usepackage{float} % Interface améliorée pour les objets flottants
\usepackage{caption} % Personnalisation des légendes
\captionsetup{%
  font=footnotesize, % Taille de police des légendes
  labelfont=bf % Police en gras pour les étiquettes (par ex., Figure 3.2)
}

% Mise en page et formatage
\usepackage[left=2cm, right=2cm, top=2cm, bottom=2cm]{geometry} % Marges du document
\usepackage{fancyhdr} % En-têtes et pieds de page personnalisés
\fancyhf{} % Supprimer tous les champs d'en-tête et de pied de page
\renewcommand{\headrulewidth}{0pt} % Supprimer la ligne horizontale dans l'en-tête
\fancyhead[RE]{\small\nouppercase\leftmark} % Page paire - titre du chapitre
\fancyhead[LO]{\small\nouppercase\rightmark} % Page impaire - titre de la section
\fancyhead[LE,RO]{\thepage} % Numéro de page sur toutes les pages
\usepackage{lastpage} % Référence au nombre de pages du document
\pagestyle{fancy}
\raggedbottom % Ne pas étirer le contenu de la page, insérer de l'espace blanc en bas de la page

% Environnements mathématiques et théorèmes
\usepackage{amsmath, amsthm, amssymb, amsfonts} % Paquets AMS pour les maths
\usepackage[framed,amsmath,thmmarks]{ntheorem} % Environnements pour les théorèmes

% Autres utilitaires
\usepackage{array,booktabs} % Gestion améliorée des tableaux
\usepackage{framed} % Encadrés pour les parties importantes
\usepackage{listings} % Formatage et mise en évidence du code source
\usepackage{courier} % Police Courier pour les listings
\usepackage{changepage} % Ajuster les marges pour des parties spécifiques de la page
\usepackage{multicol} % Colonnes multiples
\usepackage{hyperref} % Hyperliens dans le PDF
\hypersetup{%
	pdfpagelabels=true,
	plainpages=false,
	pdfauthor={Author(s)},
	pdftitle={Title},
	pdfsubject={Subject},
	bookmarksnumbered=true,
	colorlinks=true,
    breaklinks=true,
    linkcolor=black,
    citecolor=blue,
    filecolor=magenta,
    urlcolor=blue,
    linkbordercolor={1 1 1},
	pdfstartview=FitH
}

% Définition des couleurs
\definecolor{hellgelb}{rgb}{1,1,0.8}
\definecolor{colKeys}{rgb}{0,0,1}
\definecolor{colIdentifier}{rgb}{0,0,0}
\definecolor{colComments}{rgb}{0,0.5,0}
\definecolor{colString}{rgb}{0.62,0.12,0.94}
\definecolor{INSA_GM}{cmyk}{0.6,0,0,0}
\definecolor{INSA_GRIS}{cmyk}{0.7,0.6,0.5,0.3}
\definecolor{INSA_BLEU}{cmyk}{1,0.9,0.1,0}

% Commandes personnalisées
\newcommand{\insertrefproj}[1]{}
\newcommand{\refproj}[1]{\renewcommand{\insertrefproj}{\textbf{\color{INSA_GRIS}#1}}}

% Styles de page fancy
\pagestyle{fancy}

\fancypagestyle{courant}{
\fancyhf{}
\setlength{\headheight}{27pt}
\fancyhead[L]{\raisebox{-2mm}{\includegraphics[width=30mm]{Images/Logo INSA.png}}}
\fancyhead[C]{}
\fancyhead[R]{\color{INSA_GRIS}\thepage}
\fancyfoot[L]{\insertrefproj}
\fancyfoot[R]{}
\renewcommand{\headrulewidth}{0pt}
\renewcommand{\footrulewidth}{0.2pt}
}

\fancypagestyle{special}{%
\pagestyle{courant}
\fancyfoot{}
\renewcommand{\footrulewidth}{0pt}
}

\fancypagestyle{plain}{%
\fancyhf{}%
\pagestyle{courant}
}

% Formatage des titres et sections
\usepackage{titlesec}
\titleformat{\chapter}[display]{\normalfont\huge\bfseries}{\chaptertitlename\ \thechapter}{20pt}{\Huge}
\titleformat*{\section}{\normalfont\Large\bfseries}
\titleformat*{\subsection}{\normalfont\large\bfseries}
\titleformat*{\subsubsection}{\normalfont\normalsize\bfseries}

% Autres configurations et paquets
\setlength{\parindent}{0pt} % Pas d'indentation pour les paragraphes
\usepackage[contents={},color=gray]{background} % Couleur de fond
\usepackage{epic,eepic} % Positionnement pour l'image de couverture

% Réduire les espacements des titres de chapitre, trop importants par défaut
\titlespacing*{\chapter}{0pt}{-25pt}{15pt} % {alinéa}{au-dessus}{en-dessous}

% Espacement des insertions de figures
\setlength{\textfloatsep}{7pt} % Espace entre le texte et une figure flottante
\setlength{\intextsep}{7pt} % Espace entre le texte et une figure insérée
\setlength{\floatsep}{7pt} % Espace entre deux flottants
\renewcommand{\thefigure}{\arabic{figure}}
\newcommand{\figcaptionwithsource}[3]{\caption[#1\newline #2]{#1} \addtocontents{lof}{\protect\vspace{1\baselineskip}}}
\renewcommand{\listfigurename}{}
\counterwithout{figure}{chapter} % Ne pas réinitialiser le numéro des figures à chaque chapitre

\newcommand{\annexeref}[1]{Voir Annexe \ref{#1}}

% Page vide sans numéro de page
\let\origdoublepage\cleardoublepage
\newcommand{\clearemptydoublepage}{%
  \clearpage
  {\pagestyle{empty}\origdoublepage}%
}
\let\cleardoublepage\clearemptydoublepage

% Supprimer le numéro de page d'une page
\usepackage{nopageno}

% Bibliographie et index
\usepackage[nottoc]{tocbibind}

% Notes à faire
\usepackage[
  colorinlistoftodos,
  textwidth=\marginparwidth, 
  textsize=scriptsize,
]{todonotes}

% Load silence package to suppress warnings
\usepackage{silence}
% Filter out specific warnings related to everypage
\WarningFilter{everypage}{You appear to be running a version of LaTeX providing the new functionality}


% see, e.g., http://en.wikibooks.org/wiki/LaTeX/Customizing_LaTeX#New_commands
% for more information on how to create macros

%%%%%%%%%%%%%%%%%%%%%%%%%%%%%%%%%%%%%%%%%%%%%%%%
% Macros for the titlepage
%%%%%%%%%%%%%%%%%%%%%%%%%%%%%%%%%%%%%%%%%%%%%%%%
%Creates the aau titlepage
\newcommand{\customtitlepage}[3]{%
  {
    %set up various length
    \ifx\titlepageleftcolumnwidth\undefined
      \newlength{\titlepageleftcolumnwidth}
      \newlength{\titlepagerightcolumnwidth}
    \fi
    \setlength{\titlepageleftcolumnwidth}{0.5\textwidth-\tabcolsep}
    \setlength{\titlepagerightcolumnwidth}{\textwidth-2\tabcolsep-\titlepageleftcolumnwidth}
    %create title page
    \thispagestyle{empty}
    \noindent%
    \begin{tabular}{@{}ll@{}}
      \parbox{\titlepageleftcolumnwidth}{
          \includegraphics[width=\titlepageleftcolumnwidth]{Images/Logo INSA.png}
        }
      } &
      \parbox{\titlepagerightcolumnwidth}{\raggedleft\sf\small
        #2
      }\bigskip\\
       #1 &
      \parbox[t]{\titlepagerightcolumnwidth}{%
      \iflanguage{french}{%
        \textbf{Résumé:}\bigskip\par
        }{%
        \textbf{Abstract:}\bigskip\par
        }
        \fbox{\parbox{\titlepagerightcolumnwidth-2\fboxsep-2\fboxrule}{%
          #3
        }}
      }\\
    \end{tabular}
    \vfill
    \iflanguage{french}{%
      \noindent{\footnotesize\emph{Le contenu de ce rapport est librement disponible, mais sa publication (avec référence) ne peut être entreprise qu'avec l'accord de l'auteur.}}
    }{%
      \noindent{\footnotesize\emph{The content of this report is freely available, but publication (with reference) may only be pursued due to agreement with the author.}}
    }
    \clearpage
  }



%Create english project info
\newcommand{\englishprojectinfo}[7]{%
  \parbox[t]{\titlepageleftcolumnwidth}{
    \textbf{Title:}\\ #1\bigskip\par
    \textbf{Subjects:}\\ #2\bigskip\par
    \textbf{Project Period:}\\ #3\bigskip\par
    \textbf{Project Group:}\\ #4\bigskip\par
    \textbf{Participant(s):}\\ #5\bigskip\par
    \textbf{Supervisor(s):}\\ #6\bigskip\par
    \textbf{Page Numbers:} \pageref{LastPage}\bigskip\par
    \textbf{Date of Completion:}\\ #7
  }
}

\newcommand{\frenchprojectinfo}[7]{%
  \parbox[t]{\titlepageleftcolumnwidth}{
    \textbf{Titre :}\\ #1\bigskip\par
    \textbf{Matières :}\\ #2\bigskip\par
    \textbf{Période du projet :}\\ #3\bigskip\par
    \textbf{Groupe de projet :}\\ #4\bigskip\par
    \textbf{Participant(s) :}\\ #5\bigskip\par
    \textbf{Superviseur(s) :}\\ #6\bigskip\par
    \textbf{Nombre de pages :} \pageref{LastPage}\bigskip\par
    \textbf{Date de réalisation :}\\ #7
  }
}

%%%%%%%%%%%%%%%%%%%%%%%%%%%%%%%%%%%%%%%%%%%%%%%%
% An example environment
%%%%%%%%%%%%%%%%%%%%%%%%%%%%%%%%%%%%%%%%%%%%%%%%
\theoremheaderfont{\normalfont\bfseries}
\theorembodyfont{\normalfont}
\theoremstyle{break}
\def\theoremframecommand{{\color{gray!50}\vrule width 5pt \hspace{5pt}}}
\newshadedtheorem{exa}{Example}[chapter]
\newenvironment{example}[1]{%
		\begin{exa}[#1]
}{%
		\end{exa}
}


% Fix for ntheorem conflict
\makeatletter
\let\c@theorem\relax
\makeatother

% Fix for todonotes margin warning
\setlength{\marginparwidth}{2cm}

\title{MyTemplate}
\refproj{Template}
\author{Paul Fougeray}
\date{June 2024}

\begin{document}

\pdfbookmark[0]{Front page}{label:frontpage}%

\begin{titlepage}
\newgeometry{top=0cm,bottom=1.2cm,right=0cm,left=0cm}

  \backgroundsetup{
   scale=1.1,
   angle=0,
   opacity=1.0,  %% adjust
   contents={\includegraphics[width=\paperwidth,height=\paperheight]{Images/waves.pdf}}
    }
		
  \begin{center} %%please do not change the height or width of the frontpage image
    \centerline{\includegraphics[totalheight=0.5\paperwidth,width=1\paperwidth]{Images/coverpageImage.jpg}}% 
  \end{center}
	
	\vspace*{-1.1cm}
  {\noindent\color{customblue}\fboxsep0pt\colorbox{white}{\begin{tabular}{@{}p{\paperwidth}@{}}
    \centerline{
    \begin{minipage}{0.85\textwidth}
        \bigskip
				\bigskip
        \centering
        \Huge{\textbf{
          Projet Intégratif ITIneraire
        }}
    \end{minipage}
    }
		

			
	\centerline{
	\begin{minipage}{0.9\textwidth}
        \bigskip
        \centering
        {\Large
        Arrigoni Ambroise, Fougeray Paul, Sanson Dylan, Sourdrille Nathan, Zouaghi Rayan
        }
    \end{minipage}
    }
			
			
    \centerline{
    \begin{minipage}{0.9\textwidth}
        \bigskip
        \centering
%% Comment this section if you are not doing Bachelor or Master Project   
        {\Large
        ITI3 groupe 1
      %Bachelor Project
        }
        \smallskip
    \end{minipage}
    }
			
  \end{tabular}}}

  \vfill
  \begin{figure}[!b]
	\centering
    \includegraphics[width=0.2\paperwidth]{Images/Logo INSA.png}
  \end{figure}
\end{titlepage}
\restoregeometry
\thispagestyle{empty}
{\small
\strut\vfill % push the content to the bottom of the page
\noindent Copyright \copyright{} Exemple 2024\par
% \vspace{0.2cm}
% \noindent Lorem ipsum dolor sit amet, consectetur adipiscing elit. Nullam sit amet interdum tortor. Phasellus nec metus est. Nullam lacinia arcu blandit, ornare sapien sed, hendrerit leo. Cras placerat turpis id leo bibendum facilisis. Praesent non nunc ut tellus laoreet egestas tempor vel sem. Cras rhoncus purus ac aliquam auctor. Donec lacinia laoreet nisl, sit amet porttitor mauris fringilla sed. Proin volutpat varius tempor. Sed dictum risus lacinia mi bibendum, ut porttitor metus ornare. Sed mollis dolor mauris, sit amet sodales nunc venenatis varius. Vestibulum ut dolor venenatis eros imperdiet dictum.
}
\clearpage

\cleardoublepage
{\selectlanguage{french}
\pdfbookmark[0]{French title page}{label:titlepage_fr}
\customtitlepage{%
  \frenchprojectinfo{
    Projet Intégratif ITIneraire %title
  }{%
    Capteurs et Statistiques %theme
  }{%
    Mars-Mai 2025 %project period
  }{%
    Groupe 9 % project group
  }{%
    %list of group members
    Arrigoni Ambroise\\
    Fougeray Paul\\
    Sanson Dylan\\
    Sourdrille Nathan\\
    Zouaghi Rayan
  }{%
    %list of supervisors
    Condat Robin\\
    Rogozan Alexandrina
  }{%
    \today % date of completion
  }%
}{%department and address
  \textbf{Département ITI}\\
  INSA Rouen Normandie\\
  \href{https://www.insa-rouen.fr/}{https://www.insa-rouen.fr/}
}{% Résumé
    L'objectif de ce projet d'intégration est de concurrencer l'application mobile Strava,
    dédiée pour l'enregistrement des activités sportives par GPS. Pour cela, vous devrez concevoir
    et développer un système d'acquisition permettant l'estimation de trajectoire d'un parcours
    fait à pied.
}
}


% Table des matières
\tableofcontents

% Introduction
\chapter{Introduction}
\setcounter{page}{1}
\thispagestyle{courant}

Ce document présente un exemple d'utilisation de plusieurs fonctionnalités en LaTeX.

% Section avec équations mathématiques
\section{Équations mathématiques}

Voici une équation mathématique simple :
\begin{equation}
    E = mc^2 \times \displaystyle{\sum_{n=1}^N{k^n}}
\end{equation}

% Section avec une figure
\section{Insertion de figures}

Voici une figure insérée dans le texte (Figure \ref{fig:example}).

\begin{figure}[H]
    \centering
    \includegraphics[width=\textwidth]{example-image-b}
    \caption{Exemple de figure}
    \label{fig:example}
\end{figure}

\clearpage
\listoffigures
\thispagestyle{courant}

% Nouveau chapitre
\chapter{Tâches et Références}
\thispagestyle{courant}

\documentclass[12pt,a4paper]{article}

\usepackage[utf8]{inputenc}
\usepackage[T1]{fontenc}
\usepackage[french]{babel}
\usepackage{amsmath, amssymb}
\usepackage{geometry}
\usepackage{graphicx}
\usepackage{hyperref}

\geometry{margin=2.5cm}

\title{Modélisation et Estimation par Filtre de Kalman}
\author{}
\date{}

\begin{document}

\maketitle
\section*{Données disponibles}
Dans le cadre de ce projet, les données accessibles sont les suivantes :
\begin{itemize}
    \item \textbf{Position GPS :}
    \begin{itemize}
        \item $\theta$ : longitude (coordonnée est-ouest)
        \item $\lambda$ : latitude (coordonnée nord-sud)
    \end{itemize}

    \item \textbf{Données de l'IMU (Inertial Measurement Unit) :}
    \begin{itemize}
        \item \textbf{Accéléromètre :} Mesures d'accélération linéaire
        \begin{itemize}
            \item $a_x$ : accélération selon l'axe $x$
            \item $a_y$ : accélération selon l'axe $y$
            \item $a_z$ : accélération selon l'axe $z$
        \end{itemize}

        \item \textbf{Gyroscope :} Mesures de vitesse angulaire
        \begin{itemize}
            \item $\dot{\omega}_x$ : vitesse angulaire autour de l'axe $x$ (roulis)
            \item $\dot{\omega}_y$ : vitesse angulaire autour de l'axe $y$ (tangage)
            \item $\dot{\omega}_z$ : vitesse angulaire autour de l'axe $z$ (lacet)
        \end{itemize}

        \item \textbf{Magnétomètre :} Mesures du champ magnétique terrestre
        \begin{itemize}
            \item $m_x$ : composante selon l'axe $x$
            \item $m_y$ : composante selon l'axe $y$
            \item $m_z$ : composante selon l'axe $z$
        \end{itemize}
    \end{itemize}
\end{itemize}

\subsection*{Remarques}
\begin{itemize}
    \item Les données GPS ($\theta$, $\lambda$) donnent la position absolue, mais avec une précision limitée (erreur de quelques mètres).
    \item Les données de l'IMU fournissent des mesures inertielles précises à court terme, mais sujettes à une dérive temporelle (erreur qui s'accumule). De plus, l'accélération s'exprime dans le repère de l'IMU lui-même et non dans un repère terrestre.
    \item Le magnétomètre permet de s'orienter par rapport au nord magnétique, mais peut être perturbé par des interférences locales.
\end{itemize}

\section*{Premier modèle basique}
En raison de la nature instable de certains de nos composants, nous serions tentés de créer des modèles de filtre de Kalman compliqués afin de pouvoir corriger les différentes erreurs qui s'accumulent. Mais pour débuter, nous souhaitons faire un modèle simple, linéaire, qui ne nécessite pas l'utilisation d'un filtre de Kalman étendu.

\subsection*{Rappel filtre de Kalman}
En statistique et en théorie du contrôle, le filtre de Kalman est un estimateur récursif à réponse impulsionnelle infinie permettant d'estimer les états d'un système dynamique à partir de mesures bruitées ou incomplètes \cite{wikip_kalman}.

Le filtre de Kalman en contexte discret est un estimateur récursif : l'état courant est estimé à partir de l'estimation de l'état précédent et des mesures actuelles. Le filtre de Kalman suppose que le processus discret réel $\mathbf{x}_{k}$ (où $k$ dénote l'indice de temps), suit la loi d'évolution linéaire suivante :
\[
\mathbf{x}_{k} = \mathbf{F}_{k}\mathbf{x}_{k-1} + \mathbf{G}_{k}\mathbf{u}_{k} + \mathbf{w}_{k}
\]
où :
\begin{itemize}
    \item $\mathbf{F}_k$ : matrice de transition entre l'état $k-1$ et l'état $k$
    \item $\mathbf{u}_k$ : commande d'entrée
    \item $\mathbf{G}_k$ : matrice de contrôle reliant $\mathbf{u}_k$ et $\mathbf{x}_k$
    \item $\mathbf{w}_{k}$ : bruit d'évolution, gaussien centré de covariance $\mathbf{Q}_{k}$
\end{itemize}

L'observation $\mathbf{z}_k$ à l'instant $k$ est donnée par :
\[
\mathbf{z}_k = \mathbf{H}_k\mathbf{x}_k + \mathbf{v}_k
\]
avec :
\begin{itemize}
    \item $\mathbf{H}_k$ : matrice d'observation à l'instant $k$
    \item $\mathbf{v}_k$ : bruit de mesure
\end{itemize}

\subsection*{Prétraitement des données}
Maintenant que nous savons quels éléments nous devons définir ainsi que les données que nous avons à notre disposition, nous pouvons modéliser le système.
Cependant, un problème apparaît : nos acquisitions sont en coordonnées sphériques. Cela complique les traitements, qui risquent de ne pas être linéaires. Il est donc préférable de les transformer en coordonnées cartésiennes pour pouvoir appliquer un modèle linéaire, notamment pour l'intégration dans un filtre de Kalman classique.

Pour convertir les coordonnées sphériques en coordonnées cartésiennes, on utilise :
\begin{align*}
x &= R_t \cos(\lambda) \cos(\theta) \\
y &= R_t \cos(\lambda) \sin(\theta) \\
z &= R_t \sin(\lambda)
\end{align*}
avec :
\begin{itemize}
    \item $R_t$ : rayon terrestre
    \item $\lambda$ : latitude en radians
    \item $\theta$ : longitude en radians
\end{itemize}

Comme dit plus haut, les accélérations sont exprimées dans le repère local de l'accéléromètre. Pour que notre filtre fonctionne correctement, il faut les exprimer dans le repère terrestre. On applique donc une matrice de rotation à nos vecteurs d'accélération.

Les angles d'orientation sont obtenus par intégration des vitesses angulaires :
\begin{align}
    \omega_{x,k+1} &= \omega_{x,k} + \dot{\omega}_{x,k} \, \Delta t \\
    \omega_{y,k+1} &= \omega_{y,k} + \dot{\omega}_{y,k} \, \Delta t \\
    \omega_{z,k+1} &= \omega_{z,k} + \dot{\omega}_{z,k} \, \Delta t
\end{align}

Conditions initiales :
\[
\omega_{x,0}, \quad \omega_{y,0}, \quad \omega_{z,0} \quad \text{calculés à } t = 0 \text{ avec les données du magnétomètre}
\]

On en déduit la matrice de rotation $R_k$ à appliquer à nos vecteurs d'accélération à l'instant $k$ :
\[
R_k = R_{\omega_{x,k}} R_{\omega_{y,k}} R_{\omega_{z,k}}
\]

Ainsi, on obtient l'accélération dans le repère terrestre :
\[
\tilde{\mathbf{a}} =
\begin{pmatrix}
    \tilde{a}_x \\
    \tilde{a}_y \\
    \tilde{a}_z
\end{pmatrix}
=
R_k
\begin{pmatrix}
    a_x \\
    a_y \\
    a_z
\end{pmatrix}
\]

\subsection*{Modélisation du problème}
Toutes nos données sont désormais transformées pour être utilisables.

On définit le vecteur d'état $\mathbf{x}_k$ :
\[
\mathbf{x}_k =
\begin{pmatrix}
    x \\
    y \\
    z \\
    v_x \\
    v_y \\
    v_z
\end{pmatrix}
\]

À partir des équations classiques du mouvement, on a :
\[
\begin{cases}
x_{k+1} = x_k + v_{x,k} \, dt + \frac{1}{2} \tilde{a}_{x,k} \, dt^2 \\
y_{k+1} = y_k + v_{y,k} \, dt + \frac{1}{2} \tilde{a}_{y,k} \, dt^2 \\
z_{k+1} = z_k + v_{z,k} \, dt + \frac{1}{2} \tilde{a}_{z,k} \, dt^2 \\
v_{x,k+1} = v_{x,k} + \tilde{a}_{x,k} \, dt \\
v_{y,k+1} = v_{y,k} + \tilde{a}_{y,k} \, dt \\
v_{z,k+1} = v_{z,k} + \tilde{a}_{z,k} \, dt
\end{cases}
\]

Ce système est mis sous forme matricielle :
\[
\mathbf{x}_{k+1} = \mathbf{F}_k \mathbf{x}_k + \mathbf{G}_k \mathbf{u}_k + \mathbf{w}_k
\]
avec :
\[
\mathbf{F}_k =
\begin{pmatrix}
    1 & 0 & 0 & dt & 0 & 0 \\
    0 & 1 & 0 & 0 & dt & 0 \\
    0 & 0 & 1 & 0 & 0 & dt \\
    0 & 0 & 0 & 1 & 0 & 0 \\
    0 & 0 & 0 & 0 & 1 & 0 \\
    0 & 0 & 0 & 0 & 0 & 1
\end{pmatrix}
\quad \text{et} \quad
\mathbf{G}_k =
\begin{pmatrix}
    \frac{1}{2} dt^2 & 0 & 0 \\
    0 & \frac{1}{2} dt^2 & 0 \\
    0 & 0 & \frac{1}{2} dt^2 \\
    dt & 0 & 0 \\
    0 & dt & 0 \\
    0 & 0 & dt
\end{pmatrix}
\]
et $\mathbf{u}_k = \tilde{\mathbf{a}}_k$
\\
Enfin, comme on souhaite observer uniquement la position, la matrice d'observation $\mathbf{H}_k$ est :
\[
\mathbf{H}_k =
\begin{pmatrix}
    1 & 0 & 0 & 0 & 0 & 0 \\
    0 & 1 & 0 & 0 & 0 & 0 \\
    0 & 0 & 1 & 0 & 0 & 0
\end{pmatrix}
\]

\section*{Analyse des résultats}

Après implémentation et exécution du filtre de Kalman sur les données disponibles, nous pouvons analyser les résultats obtenus afin d’évaluer la performance et la pertinence de notre modèle basique.

\subsection*{Validation de l’estimation de la position}

Le filtre de Kalman permet d’obtenir une estimation optimisée de la position en combinant la mesure GPS, précise à long terme mais bruitée, avec les mesures inertielle issues de l’IMU, précises à court terme mais sujettes à dérive.

\begin{itemize}
    \item \textbf{Précision spatiale :}
    La comparaison des positions estimées avec les mesures GPS brutes montre généralement une réduction du bruit aléatoire, traduisant un meilleur lissage des trajectoires.
    \item \textbf{Réactivité :}
    Grâce à l’intégration des accélérations corrigées, le filtre peut suivre les variations rapides de mouvement même en l’absence temporaire de signaux GPS (ex. dans des tunnels ou zones d’interférences).
    \item \textbf{Erreur d’estimation :}
    L’erreur quadratique moyenne (RMSE) entre la position estimée et les données GPS de référence peut être calculée pour quantifier l’amélioration obtenue par rapport aux mesures brutes.
\end{itemize}

\subsection*{Limites du modèle basique}

Malgré ces améliorations, certaines limitations inhérentes au modèle simplifié apparaissent :

\begin{itemize}
    \item \textbf{Approximation linéaire :}
    Le modèle linéaire ne prend pas en compte les non-linéarités dues à la dynamique réelle du véhicule ou des capteurs, ce qui peut dégrader les performances en cas de mouvements complexes.
    \item \textbf{Modélisation des bruits :}
    Les matrices de covariance $\mathbf{Q}_k$ (processus) et $\mathbf{R}_k$ (mesure) doivent être choisies avec soin. Un mauvais calibrage peut engendrer une mauvaise estimation, oscillante ou trop lente à réagir.
    \item \textbf{Dérive gyroscopique et erreurs d’orientation :}
    L’intégration des vitesses angulaires pour calculer la matrice de rotation $R_k$ accumule les erreurs au fil du temps, impactant la qualité de la transformation des accélérations et donc l’estimation des vitesses et positions.
    \item \textbf{Perturbations du magnétomètre :}
    Les interférences magnétiques locales peuvent fausser l’orientation initiale et rendre instable la transformation vers le repère terrestre.
\end{itemize}

\subsection*{Perspectives d’amélioration}

Pour pallier ces limites, plusieurs axes d’amélioration peuvent être envisagés :

\begin{itemize}
    \item Utilisation d’un \textbf{filtre de Kalman étendu (EKF)} ou \textbf{filtre de Kalman unscented (UKF)} afin de mieux gérer les non-linéarités dans la dynamique et l’observation.
    \item Intégration d’un modèle plus complet de la dynamique du système, par exemple en incluant les angles d’orientation comme variables d’état, pour mieux estimer la matrice de rotation.
    \item Calibration plus fine des matrices de bruit et étude des modèles statistiques des erreurs spécifiques aux capteurs.
    \item Fusion avec d’autres capteurs, comme un baromètre pour la hauteur, ou des systèmes de correction GPS différentiel pour réduire les erreurs absolues.
\end{itemize}

\subsection*{Conclusion}

Ce premier modèle linéaire simple de filtre de Kalman permet d’établir une base robuste pour la fusion des données GPS et IMU. Il montre une amélioration sensible de la qualité d’estimation de la position par rapport aux mesures brutes. Cependant, les limites relevées invitent à développer un modèle plus complet et adapté aux spécificités des capteurs et de la dynamique étudiée pour une application finale plus précise et fiable.



\bibliographystyle{plain}
\bibliography{refs}

\end{document}


% Section avec une liste de tâches
\section{Liste de tâches}

\begin{figure}[H]
    \centering
    \includegraphics[width=0.5\textwidth]{example-image-c}
    \caption{Exemple de figure}
    \label{fig:example2}
\end{figure}

Voici quelques tâches à accomplir :
\begin{itemize}
    \item Réviser la section sur les équations mathématiques.
    \item Ajouter des références bibliographiques.
    \item Finaliser la mise en page des figures.
\end{itemize}

% Section avec une citation et des références
\section{Citations et références}

Selon Einstein \cite{einstein1905}, l'équation $E = mc^2$ est fondamentale.
Ce document présente un exemple d'utilisation de plusieurs fonctionnalités en LaTeX.\footnote{Une note de bas de page ajoutée pour illustrer l'utilisation des notes de bas de page.}

\begin{thebibliography}{9}
    \thispagestyle{courant}
    \bibitem{einstein1905}
    Albert Einstein.
    \textit{Zur Elektrodynamik bewegter Körper}.
    Annalen der Physik, 1905.

    \bibitem{kalmanPDF}
    Stéphane Calderon, Floris Chabert.
    \textit{Filtrage de Kalman appliqué à une centrale inertielle multi-capteurs},
    École Nationale Supérieure des Télécommunications, Paris.
    \textit{Disponible à:} \url{https://igns.wdfiles.com/local--files/filtre-de-kalman/Kalman.pdf},
    (Consulté le 01/04/2025).

    \bibitem{wikip_jacobienne}
    \textit{Matrice jacobienne},
    Wikipédia, l'encyclopédie libre.
    \textit{Disponible à:} \url{https://fr.wikipedia.org/wiki/Matrice_jacobienne},
    (Consulté le 01/04/2025).

    \bibitem{wikip_kalman}
    \textit{Filtre de Kalman},
    Wikipédia, l'encyclopédie libre.
    \textit{Disponible à:} \url{https://fr.wikipedia.org/wiki/Filtre_de_Kalman},
    (Consulté le 01/04/2025).
\end{thebibliography}

% Annexes
\appendix
\chapter{Annexe A}
\thispagestyle{courant}

Contenu de l'annexe A.

\end{document}
