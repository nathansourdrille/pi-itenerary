\chapter{Présentation des capteurs à notre disposition}
Cette partie a pour but de présenter les capteurs que nous avons utilisés lors des travaux pratiques.

\section{Description et fonctionnement}

\subsection{GPS}
Le capteur GPS (Global Positioning System) permet d'obtenir la position géographique d’un objet en temps réel à l’aide de signaux satellites. Il fournit typiquement des données de latitude, longitude, altitude, vitesse et temps. Ce capteur est essentiel pour le suivi de trajectoire.

\subsection{Accéléromètre}
L'accéléromètre mesure les accélérations linéaires suivant les trois axes X, Y et Z. Il permet de détecter les mouvements, les chocs, et d’estimer l’orientation d’un objet (par exemple, inclinaison) lorsqu’il est combiné avec d’autres capteurs.

\subsection{Gyroscope}
Le gyroscope mesure la vitesse angulaire de rotation autour des trois axes. Il est souvent utilisé pour suivre l’orientation d’un objet en mouvement, en particulier dans les systèmes d’asservissement ou de navigation inertielle.

\subsection{Magnétomètre}
Le magnétomètre détecte le champ magnétique terrestre et fonctionne comme une boussole numérique. Il permet de déterminer l’orientation absolue par rapport au nord magnétique. Il est souvent combiné au gyroscope pour améliorer la précision de l’orientation.

\section{Avantages et inconvénients}
Chaque capteur présente des avantages et des limites :
\begin{itemize}
    \item \textbf{GPS} : Bonne précision sur la position à grande échelle, mais latence élevée et faible fréquence d’échantillonnage. Inefficace en intérieur.
    \item \textbf{Accéléromètre} : Précis pour les mouvements rapides, mais sensible au bruit et aux erreurs d’intégration.
    \item \textbf{Gyroscope} : Très réactif pour détecter les rotations, mais dérive dans le temps sans recalibrage.
    \item \textbf{Magnétomètre} : Utile pour connaître l’orientation absolue, mais perturbé par les champs magnétiques ambiants.
\end{itemize}

\section{Sources de bruitages et/ou défauts techniques}
Les capteurs sont sujets à différents types de bruits :
\begin{itemize}
    \item \textbf{GPS} : erreurs dues à la météo, aux obstacles (bâtiments), et aux multipaths.
    \item \textbf{Accéléromètre et gyroscope} : bruit thermique, erreurs de biais, et dérive.
    \item \textbf{Magnétomètre} : perturbations électromagnétiques, présence de métaux ferromagnétiques proches.
\end{itemize}

\section{Calibration effectuée}

\subsection{Démarche}
La calibration vise à réduire les erreurs systématiques. Pour l’accéléromètre et le gyroscope, nous avons mesuré les valeurs à l’arrêt pour corriger les biais. 

\subsection{Expérimentations}
Nous avons relevé les valeurs brutes dans différentes orientations et positions. Les écarts ont permis d’ajuster les offsets et les gains.

\subsection{Résultats}
Les données corrigées présentent une réduction notable du biais et une meilleure cohérence des mesures, notamment lors des déplacements rectilignes ou rotations contrôlées.

\section{Choix des paramètres d'acquisition}
Les paramètres ont été choisis pour équilibrer précision et consommation :
\begin{itemize}
    \item GPS : 1 Hz, suffisant pour les déplacements lents.
    \item Accéléromètre et gyroscope : 50 Hz pour bien suivre les mouvements.
    \item Magnétomètre : 10 Hz, suffisant pour détecter l’orientation.
\end{itemize}

\section{Filtrage des signaux bruts}

Pour filtrer nos données, nous avons mis en place le filtre de Kalman.

\subsection{Présentation du filtre de Kalman}

Le filtre de Kalman est un algorithme récursif d’estimation d’état, largement utilisé dans les systèmes de navigation, de robotique et de traitement du signal. Il permet de combiner les mesures provenant de différents capteurs tout en tenant compte de leurs incertitudes respectives, afin d’obtenir une estimation plus fiable et plus précise de la variable d’intérêt (position, vitesse, orientation, etc.).

Le principe du filtre de Kalman repose sur deux étapes principales :
\begin{itemize}
    \item \textbf{Prédiction} : à partir du modèle dynamique du système, le filtre prédit l’état futur et son incertitude.
    \item \textbf{Mise à jour (correction)} : lorsque de nouvelles mesures sont disponibles, le filtre corrige son estimation en fonction de l’erreur observée entre la prédiction et la mesure réelle.
\end{itemize}

Mathématiquement, le filtre de Kalman repose sur l'hypothèse que les erreurs de mesure et les incertitudes sont de nature gaussienne (bruit blanc) et que le système peut être modélisé de manière linéaire. Pour des systèmes non linéaires, des variantes comme le filtre de Kalman étendu (EKF) ou le filtre de Kalman non linéaire (UKF) sont utilisées.

\subsection{Application dans notre cas}

Dans notre projet, le filtre de Kalman a été utilisé pour fusionner les données de l'accéléromètre et du gyroscope. L’accéléromètre fournit une estimation de l’orientation à long terme mais bruitée, tandis que le gyroscope offre des mesures plus stables à court terme mais sujettes à la dérive. Le filtre de Kalman permet de combiner ces deux sources d'information pour obtenir une estimation de l’orientation à la fois stable et précise.

Cette fusion permet de limiter les effets des bruits haute fréquence de l’accéléromètre et de compenser la dérive du gyroscope, ce qui est essentiel pour une utilisation fiable des capteurs dans un système embarqué.



