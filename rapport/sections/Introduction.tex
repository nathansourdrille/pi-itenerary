% Introduction
\chapter{Introduction}

L’objectif principal de ce projet est de concevoir un capteur de position destiné à une application de type STRAVA, dédiée au suivi et à l’analyse des activités sportives en extérieur. Ce dispositif sera constitué de plusieurs capteurs intégrés, notamment un GPS, un accéléromètre, un gyroscope et un magnétomètre. L’intégration de ces différents capteurs vise à fournir des données précises et fiables sur la position, le mouvement et l’orientation de l’utilisateur.

Ce projet s’inscrit dans le cadre de notre formation et constitue une mise en pratique concrète des connaissances théoriques acquises en cours, notamment en matière de capteurs. Il offre également l’opportunité d’appliquer des outils mathématiques, en particulier des méthodes statistiques avancées. À ce titre, l’un des axes majeurs de notre travail sera l’utilisation du filtre de Kalman, un algorithme de filtrage permettant de fusionner efficacement les données issues de capteurs multiples, tout en tenant compte de leur incertitude respective.

Au-delà des aspects techniques, ce projet mettra en valeur le travail en équipe, la répartition des tâches, la collaboration entre membres et la capacité à mener un projet technique de bout en bout. Il s’agit donc d’un exercice complet, à la fois technique, méthodologique et collaboratif, nous préparant à des projets similaires dans un cadre professionnel.
