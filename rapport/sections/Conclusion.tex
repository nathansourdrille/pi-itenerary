\chapter*{Conclusion}

Ce projet nous a permis de concevoir et de réaliser un système complet d’estimation de trajectoire basé sur la fusion de données capteurs, avec pour objectif de proposer une alternative simplifiée à des applications comme Strava.

Nous avons étudié, calibré et synchronisé différents capteurs (GPS, accéléromètre, gyroscope, magnétomètre), tout en développant une application capable d’en exploiter les mesures en temps réel. L’intégration d’un filtre de Kalman a constitué une étape centrale, permettant de combiner efficacement les données GPS et IMU afin d’améliorer la précision de localisation.

Nous avons également enrichi notre système d’une fonctionnalité de course d’orientation, rendant l’usage plus ludique et interactif. Sur le plan méthodologique, nous avons mis en œuvre des outils statistiques rigoureux pour optimiser les paramètres d’acquisition, détecter les valeurs aberrantes et évaluer les performances du système.

Malgré quelques limitations dans la validation expérimentale, dues à un manque de données de référence sur certaines acquisitions, les résultats obtenus sont prometteurs. Ce travail ouvre la voie à des perspectives d’amélioration, notamment en explorant des variantes non linéaires du filtre de Kalman, en affinant les modèles dynamiques ou en intégrant d’autres sources de données.

Enfin, ce projet nous a permis de mobiliser un large éventail de compétences — théoriques, pratiques et collaboratives — tout en nous confrontant à une problématique proche d’un contexte réel d’ingénierie. Il constitue une expérience formatrice riche, tant sur le plan technique que dans la gestion de projet en équipe.

