\chapter{Analyse des performances du système d'acquisition}
\section{Méthodologie pour la détermination du paramétrage du système d'acquisition}
\section{Description et analyse du choix des méthodes statistiques avec les résultats obtenus}
\section{Analyse des résultats de tracking obtenus}

\subsection{Mouvement extérieur 1}

\subsubsection{Comparaison des trajets}
\begin{figure}[H]
    \centering
    \includegraphics[height=400pt, width = 500pt]{../traitements/mouvement_exterieur1/comparaison_trajets}
    \caption{Comparaison trajet mouvement extérieur 1 (Gps vs Téléphone)}
    \label{fig: Comparaison trajet mouvement exterieur 1}
\end{figure}

Dans cette exemple, on a fait deux tours à l'extérieur et on remarque qu'en prenant les données du téléphone comme repère. Le GPS qu'on utilise n'est pas
trop mauvais (FIXEZ çA JE DIS DE LA MERDE).

\subsubsection{Comparaison latitude  et longitude}
\begin{figure}[h]
    \centering
    \begin{minipage}{0.45\textwidth}
        \centering
        \includegraphics[height = 250pt, width=\linewidth]{../traitements/mouvement_exterieur1/comparaison_latitude}
        \caption{ Comparaison latitude (Gps vs Téléphone) }
    \end{minipage}
    \hfill
    \begin{minipage}{0.45\textwidth}
        \centering
        \includegraphics[height = 250pt , width=\linewidth]{../traitements/mouvement_exterieur1/comparaison_longitude}
        \caption{ Comparaison longitude (Gps vs Téléphone) mouvement exterieur1}
    \end{minipage}
\end{figure}

\subsubsection{Boite à moustache vitesse / accéleration}

\begin{figure}[H]
    \centering
    \includegraphics[height=250pt, width = \linewidth]{../traitements/mouvement_exterieur1/boxplots_vitesse_acceleration}
    \caption{Boite à moustache }
    \label{fig: Boxplot mouvement_exterieur1}
\end{figure}


\subsubsection{Evolution de l'erreur mouvement exterieur1}

\begin{figure}[H]
    \centering
    \includegraphics[height=300pt, width = 400pt]{../traitements/mouvement_exterieur1/erreur_evolution}
    \caption{ Evolution de l'erreur en cours du temps }
    \label{fig: erreur_evolution  mouvement_exterieur1}
\end{figure}


\subsubsection{Valeurs aberrantes}

\begin{figure}[H]
    \centering
    \includegraphics[height=250pt, width = \linewidth]{../traitements/mouvement_exterieur1/valeurs_aberrantes}
    \caption{ Valeurs aberrantes}
    \label{fig: Valeurs aberrantes mouvement_exterieur1}
\end{figure}






\section{Analyse des résultats obtenus en cas de défaillance capteur}
\subsection{Acquisition immobile exterieur 1}
\subsubsection{Comparaison trajet Gps vs Téléphone}
\begin{figure}[H]
    \centering
    \includegraphics[height=400pt, width = 500pt]{../traitements/immobile_exterieur1/comparaison_trajets}
    \caption{Comparaison trajet immobile exterieur 1 (Gps vs Téléphone)}
    \label{fig: Comparaison trajet immobile exterieur 1}
\end{figure}
On remarque que même à l'extérieur, les acquisitions du gps et du téléphone ne sont pas assez précises. Le trajet nous indique qu'on était en mouvement même
si on était immobile.

\subsubsection{Comparaison latitude  et longitude}
\begin{figure}[h]
    \centering
    \begin{minipage}{0.45\textwidth}
        \centering
        \includegraphics[height = 250pt, width=\linewidth]{../traitements/immobile_exterieur1/comparaison_latitude}
        \caption{ Comparaison latitude (Gps vs Téléphone) }
    \end{minipage}
    \hfill
    \begin{minipage}{0.45\textwidth}
        \centering
        \includegraphics[height = 250pt , width=\linewidth]{../traitements/immobile_exterieur1/comparaison_longitude}
        \caption{ Comparaison longitude (Gps vs Téléphone)}
    \end{minipage}
\end{figure}

\subsubsection{Boite à moustache vitesse / accéleration}

\begin{figure}[H]
    \centering
    \includegraphics[height=250pt, width = \linewidth]{../traitements/immobile_exterieur1/boxplots_vitesse_acceleration}
    \caption{Boite à moustache }
    \label{fig: Boxplot immobile exterieur 1}
\end{figure}


\subsubsection{Evolution de l'erreur }

\begin{figure}[H]
    \centering
    \includegraphics[height=300pt, width = 400pt]{../traitements/immobile_exterieur1/erreur_evolution}
    \caption{ Evolution de l'erreur en cours du temps }
    \label{fig: erreur_evolution immobile exterieur }
\end{figure}


\subsubsection{Valeurs aberrantes}

\begin{figure}[H]
    \centering
    \includegraphics[height=250pt, width = \linewidth]{../traitements/immobile_exterieur1/valeurs_aberrantes}
    \caption{ Valeurs aberrantes }
    \label{fig: Valeurs aberrantes immobile exterieur}
\end{figure}

\subsection{ Acquisition immobile exterieur 2 }

\begin{figure}[H]
    \centering
    \includegraphics[height=400pt, width = 500pt]{../traitements/immobile_exterieur2/comparaison_trajets}
    \caption{Comparaison trajet immobile exterieur 2 (Gps vs Téléphone)}
    \label{fig: Comparaison trajet immobile exterieur 2}
\end{figure}

Ici, on peut remarquer que les données fournies par le téléphone sont beaucoup plus précises que les données fournies. En effet, on ne peut pas se fier
aux données du GPS. (à conclure après je sais pas voir avec la démarche etc)


subsubsection{Comparaison latitude  et longitude}
\begin{figure}[h]
    \centering
    \begin{minipage}{0.45\textwidth}
        \centering
        \includegraphics[height = 250pt, width=\linewidth]{../traitements/immobile_exterieur2/comparaison_latitude}
        \caption{ Comparaison latitude (Gps vs Téléphone) immobile exterieur 2}
    \end{minipage}
    \hfill
    \begin{minipage}{0.45\textwidth}
        \centering
        \includegraphics[height = 250pt , width=\linewidth]{../traitements/immobile_exterieur2/comparaison_longitude}
        \caption{ Comparaison longitude (Gps vs Téléphone) immobile exterieur 2 }
    \end{minipage}
\end{figure}

\subsubsection{Boite à moustache vitesse / accéleration}

\begin{figure}[H]
    \centering
    \includegraphics[height=250pt, width = \linewidth]{../traitements/immobile_exterieur2/boxplots_vitesse_acceleration}
    \caption{Boite à moustache }
    \label{fig: Boxplot immobile exterieur 2}
\end{figure}


\subsubsection{Evolution de l'erreur }

\begin{figure}[H]
    \centering
    \includegraphics[height=300pt, width = 400pt]{../traitements/immobile_exterieur2/erreur_evolution}
    \caption{ Evolution de l'erreur en cours du temps }
    \label{fig: erreur_evolution immobile exterieur 2 }
\end{figure}


\subsubsection{Valeurs aberrantes}

\begin{figure}[H]
    \centering
    \includegraphics[height=250pt, width = \linewidth]{../traitements/immobile_exterieur2/valeurs_aberrantes}
    \caption{ Valeurs aberrantes }
    \label{fig: Valeurs aberrantes immobile exterieur 2}
\end{figure}
