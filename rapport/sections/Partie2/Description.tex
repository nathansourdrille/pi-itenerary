\chapter{Présentation des capteurs à notre disposition}
Cette partie a pour but de présenter les capteurs que nous avons utilisés lors des travaux pratiques.

\section{Description et fonctionnement}

\subsection{GPS}
Le capteur GPS (Global Positioning System) est un composant permettant de capter les signaux envoyés par les satellites GPS afin de déterminer une position géographique précise. Pour rappel, le GPS est le nom du système GNSS (Global Navigation Satellite System) développé par les États-Unis. Il s'agit d’un système de positionnement basé sur des satellites artificiels placés en orbite et fonctionnant de manière coordonnée.
\vspace{0.5cm}\\
Ce réseau repose sur une constellation d’environ 30 satellites opérationnels, situés à une altitude d’environ 20 200 km. La précision pour les usages civils varie généralement entre 3 et 5 mètres. Plus le nombre de satellites captés est élevé, plus la précision augmente, notamment grâce à la correction des erreurs liées à l’atmosphère ou aux interférences.
\vspace{0.5cm}\\
Le principe de fonctionnement est plutôt simple: chaque satellite émet en continu un signal contenant l'heure exacte d’émission et sa position dans l’espace, transmis à la vitesse de la lumière. À la réception, le capteur GPS calcule sa distance à chaque satellite en mesurant le temps de parcours du signal.
\vspace{0.5cm}\\
Pour fonctionner correctement, le capteur doit capter les signaux d’au moins quatre satellites : les signaux de trois d’entre eux permettent de déterminer une position en deux dimensions via la trilatération, tandis que le quatrième sert à corriger l’erreur d’horloge du récepteur et à déterminer l’altitude.

Il fournit typiquement des données de latitude, longitude, altitude, vitesse et temps. Ce capteur est essentiel pour le suivi de trajectoire.

\subsection{Accéléromètre}
Un accéléromètre est un dispositif qui mesure l'accélération (ou le mouvement) auquel il est soumis en appliquant le principe fondamentale de la dynamique: \\
\begin{adjustwidth}{7.5cm}{0cm} $\vec{F} = m \vec{a}$.
\end{adjustwidth}
\vspace{0.2cm}\\
En termes simples, l'accélèromètre détecte les changements de vitesse ou de direction. L'accéléromètre fonctionne en mesurant la force exercée sur un petit composant interne, souvent une masse suspendue à un ressort ou un capteur piézoélectrique. Cette masse se déplace en fonction des variations d'accélération de l'objet. Plus l'objet accélère, plus cette masse se déplace. Ce mouvement est ensuite mesuré par le capteur.
\vspace{0.5cm}\\
L'accéléromètre mesure les accélérations linéaires le long des trois axes (X, Y et Z). Il permet ainsi de détecter les mouvements, les chocs, et d’estimer l’orientation d’un objet (comme l'inclinaison). Lorsqu'il est combiné avec d’autres capteurs, il peut aussi fournir des informations plus précises sur la position et le mouvement de l'objet dans l'espace.

\subsection{Gyroscope}
Le gyroscope est un capteur qui mesure la vitesse angulaire, c’est-à-dire la rapidité avec laquelle un objet tourne autour de ses axes (X, Y ou Z). Contrairement à l’accéléromètre qui détecte des mouvements linéaires, le gyroscope mesure les rotations. Le principe de fonctionnement du gyroscope repose sur la conservation du moment cinétique. Lorsqu’un rotor (ou masse tournante) est mis en rotation, il tend à conserver son orientation dans l’espace, conformément à la loi :\\
\begin{adjustwidth}{6cm}{0cm}$\dfrac{d\vec{L}_O}{dt} = 0 \Rightarrow \vec{L}_O = \vec{\text{Cte}}$
\end{adjustwidth}
\vspace{0.1cm}\\
avec $\vec{L}_O$ le moment cinétique par rapport à un point O.
\vspace{0.5cm}\\
Dans les gyroscopes MEMS (Micro-Electro-Mechanical Systems) modernes, cette rotation est généralement mesurée par effet Coriolis. Une petite structure oscillante à l’intérieur du capteur subit une déviation lorsqu’elle est en mouvement de rotation. Cette déviation est proportionnelle à la vitesse angulaire et peut être détectée électriquement.
\vspace{0.5cm}\\
Grâce à ces mesures, le gyroscope est capable de suivre les changements d’orientation d’un appareil, ce qui le rend essentiel pour notre processus de développement.

\subsection{Magnétomètre}
Le magnétomètre est un capteur qui mesure l’intensité et la direction d’un champ magnétique, généralement celui de la Terre. Il fonctionne comme une boussole numérique, permettant de déterminer l’orientation absolue d’un objet par rapport au nord magnétique. Contrairement au gyroscope, qui mesure des vitesses de rotation relatives, le magnétomètre fournit un repère fixe dans l’espace.
\vspace{0.5cm}\\
Le principe de fonctionnement le plus courant repose sur l’effet Hall. Lorsqu’un courant électrique traverse un conducteur ou un semi-conducteur soumis à un champ magnétique perpendiculaire, une différence de potentiel apparaît sur les côtés du matériau. Cette tension, appelée tension de Hall, est proportionnelle à l’intensité du champ magnétique traversant le capteur. Elle est mesurée électroniquement pour en déduire la composante du champ magnétique selon chaque axe (X, Y, Z).
\vspace{0.5cm}\\
Le magnétomètre permet donc d’obtenir une mesure vectorielle du champ magnétique terrestre, et ainsi de calculer l’orientation d’un objet par rapport aux points cardinaux. Toutefois, il est sensible aux perturbations électromagnétiques de l’environnement (matériaux ferromagnétiques, courants, appareils électroniques, etc.). Pour cette raison, il est souvent combiné avec un gyroscope et un accéléromètre, permettant une estimation plus fiable et précise de l’orientation dans l’espace.

\section{Avantages et inconvénients}
Chaque capteur mesure une grandeur physique, mais aucun capteur n'est parfait. \\
Chacun présente des atouts et des limitations qu'il convient de connaitre afin de concevoir un système de mesure fiable.
\begin{itemize}
    \item \textbf{GPS} : \begin{itemize}
        \item \textit{Avantages} : \\ Le GPS a une bonne précision géographique (de l'ordre de quelques mètres pour un GPS civil).\\ De plus, il fonctionne de manière autonome. En effet, il n'a pas
        besoin d'infrastructure locale. En outre, il permet de déterminer la position absolue sur le globe (latitude, longitude, altitude).
        \item \textit{Limites} : \\ Même si le GPS nous offre plusieurs avantages. Ce dernier a quelques limites à ne pas ignorer. En fait, il a une latence élevée de l'ordre de plusieurs centaines de millisecondes à quelques secondes.\\
        Le GPS a aussi une faible fréquence d'échantillonnage (entre 1Hz et 10Hz), ce qui n'est parfois pas suffisant surtout pour capter les dynamiques rapides. Il ne faut pas oublier aussi qu'il est peu fiable
        en environnements clos comme les bâtiments, tunnels, etc... car il nécessite une ligne de vue avec les satellites. Finalement, le GPS est vulnérable aux interférences radio et au brouillage.
    \end{itemize}
    \item \textbf{Accéléromètre} \begin{itemize}
        \item \textit{Avantages} : \\ L'accélèromètre nous offre une mesure directe de l'accélération linéaire dans les 3 axes (x,y,z). En effet, ce dernier est très utile pour détecter les chocs, les mouvements  ou les vibrations.
        Sa haute fréquence d'échantillonnage (plusieurs KHz) lui permet d'avoir un suivi précis des changements rapides. Les avantages les plus importants de l'accélèromètre sont sa taille et son coût. En effet, il est peu coûteux et très compact.

        \item \textit{Limites} : \\ Tout comme le GPS, l'accélèromètre a ses inconvénients.En fait, un accélèromètre est très sensible au bruit (bruit thermique, vibrations parasites).
        Les mesures nécessitent souvent un filtrage notamment : un filtre passe-bas ou filtre de Kalman. De plus, un accélèromètre ne donne pas la position absolue. Il donne seulement des informations relatives au mouvement.

    \end{itemize}
    \item \textbf{Gyroscope} : 
    \begin{itemize}
        \item \textit{Avantages} : \\ Le gyroscope permet de mesurer la vitesse angulaire, c’est-à-dire les rotations autour des axes. \\ Il est très réactif et précis à court terme, ce qui le rend particulièrement utile pour détecter des changements rapides d’orientation. \\ Grâce à sa haute fréquence d’échantillonnage, il est capable de suivre les mouvements avec une grande précision en temps réel.
        
        \item \textit{Limites} : \\ Cependant, le gyroscope présente aussi certaines limites. \\ Il souffre notamment d’un phénomène appelé dérive : les petites erreurs de mesure s’accumulent avec le temps, ce qui dégrade la précision à long terme si le capteur n’est pas recalibré. \\ De plus, il consomme généralement plus d’énergie que d’autres capteurs comme l’accéléromètre, ce qui peut poser problème dans les systèmes autonomes. 
    \end{itemize}
    
    \item \textbf{Magnétomètre} : 
    \begin{itemize}
        \item \textit{Avantages} : \\ Le magnétomètre mesure le champ magnétique terrestre, ce qui permet d’estimer l’orientation absolue par rapport au nord magnétique. \\ Contrairement au gyroscope, il ne dérive pas avec le temps. \\ Il est donc particulièrement utile pour corriger les erreurs d’orientation dans les systèmes de navigation. 
        
        \item \textit{Limites} : \\ Le principal inconvénient du magnétomètre est sa grande sensibilité aux perturbations magnétiques locales. \\ En présence de métaux ferromagnétiques ou de champs électromagnétiques générés par des appareils électroniques, ses mesures peuvent devenir très imprécises. \\ Il nécessite souvent un étalonnage régulier pour fonctionner correctement, notamment en environnement intérieur ou urbain dense.
    \end{itemize}

\end{itemize}

\section{Sources de bruitages et/ou défauts techniques}
Les capteurs sont sujets à différents types de bruits :
\item \textbf{Sources de bruit des capteurs} :
\begin{itemize}
    \item \textit{GPS} : \\ Bien que le GPS soit un outil de localisation très répandu, il reste vulnérable à plusieurs types de bruits et d’erreurs. \\ L’une des principales sources de perturbation provient des conditions atmosphériques : les variations de la densité de la troposphère (couche inférieure de l’atmosphère) et les irrégularités de l’ionosphère (couche chargée en particules ionisées) peuvent ralentir ou dévier les signaux émis par les satellites, introduisant des erreurs de positionnement. \\ Par ailleurs, en milieu urbain ou montagneux, les bâtiments et obstacles naturels bloquent la ligne de vue entre les satellites et le récepteur, causant des pertes au niveau du signal ou un manque de satellites visibles, ce qui réduit la précision. \\ Le phénomène de \textit{multipath}, où les signaux rebondissent sur des surfaces avant d’atteindre le récepteur, introduit des délais artificiels dans la réception des signaux, faussant ainsi le calcul de la distance satellite-récepteur. \\ 

    \item \textit{Accéléromètre et gyroscope} : \\ Les capteurs inertiels sont particulièrement sensibles aux erreurs internes liées à leur structure physique et électronique. \\ Le \textit{bruit thermique} est l’un des plus répandus : il résulte de l’agitation thermique des électrons dans les circuits du capteur, créant des fluctuations aléatoires dans les mesures. \\ Ces capteurs présentent également des \textit{biais}, c’est-à-dire une erreur constante (ou lentement variable) dans la mesure, qui peut varier en fonction de la température ou de l’usure du composant. \\ Lorsqu’on intègre les données de l’accéléromètre (pour obtenir la vitesse) ou du gyroscope (pour obtenir l’orientation), ces biais s’accumulent dans le temps et produisent une \textit{dérive} significative : même en l’absence de mouvement, le système peut croire qu’un déplacement ou une rotation est en cours. \\ À cela s’ajoutent les erreurs d’alignement des axes (erreurs de montage) et les vibrations mécaniques indésirables. \\ Ces limitations rendent l’utilisation isolée de ces capteurs peu fiable pour les estimations sur le long terme sans recalibrage ou fusion de capteurs.

    \item \textit{Magnétomètre} : \\ Le magnétomètre mesure le vecteur du champ magnétique terrestre, ce qui en fait un capteur clé pour estimer le cap absolu (direction nord-sud). \\ Cependant, sa fiabilité est fortement dépendante de l’environnement dans lequel il est utilisé. \\ Tout d’abord, la proximité d’objets contenant des matériaux \textit{ferromagnétiques} (comme les structures métalliques, les véhicules, ou même des meubles) peut déformer localement le champ magnétique, provoquant une erreur d’orientation. \\ Ensuite, les sources \textit{électromagnétiques} actives — moteurs, transformateurs, fils conducteurs parcourus par du courant — génèrent des champs parasites susceptibles d’invalider complètement la mesure. \\ En milieu intérieur, ces perturbations sont quasiment omniprésentes. \\ Pour cette raison, il est essentiel de procéder à un \textit{étalonnage} régulier, généralement en réalisant des mouvements circulaires pour cartographier les erreurs et appliquer des corrections logicielles. \\ Malgré cela, le magnétomètre reste peu fiable dans certains environnements, et son usage seul pour l’orientation est déconseillé sans assistance d’autres capteurs.
\end{itemize}

