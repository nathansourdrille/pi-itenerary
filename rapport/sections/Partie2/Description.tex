\chapter{Présentation des capteurs à notre disposition}
Cette partie a pour but de présenter les capteurs que nous avons utilisés lors des travaux pratiques.

\section{Description et fonctionnement}

\subsection{GPS}
Le capteur GPS (Global Positioning System) est un composant permettant de capter les signaux envoyés par les satellites GPS afin de déterminer une position géographique précise. Pour rappel, le GPS est le nom du système GNSS (Global Navigation Satellite System) développé par les États-Unis. Il s'agit d’un système de positionnement basé sur des satellites artificiels placés en orbite et fonctionnant de manière coordonnée.
\vspace{0.5cm}\\
Ce réseau repose sur une constellation d’environ 30 satellites opérationnels, situés à une altitude d’environ 20 200 km. La précision pour les usages civils varie généralement entre 3 et 5 mètres. Plus le nombre de satellites captés est élevé, plus la précision augmente, notamment grâce à la correction des erreurs liées à l’atmosphère ou aux interférences.
\vspace{0.5cm}\\
Il fournit typiquement des données de latitude, longitude, altitude, vitesse et temps. Ce capteur est essentiel pour le suivi de trajectoire.

\subsection{Accéléromètre}
Un accéléromètre est un dispositif qui mesure l'accélération (ou le mouvement) auquel il est soumis en appliquant le principe fondamentale de la dynamique: \\
\begin{adjustwidth}{7.5cm}{0cm} $\vec{F} = m \vec{a}$.
\end{adjustwidth}
\vspace{0.2cm}\\
En termes simples, l'accélèromètre détecte les changements de vitesse ou de direction. L'accéléromètre fonctionne en mesurant la force exercée sur un petit composant interne, souvent une masse suspendue à un ressort ou un capteur piézoélectrique. Cette masse se déplace en fonction des variations d'accélération de l'objet. Plus l'objet accélère, plus cette masse se déplace. Ce mouvement est ensuite mesuré par le capteur.
\vspace{0.5cm}\\
L'accéléromètre mesure les accélérations linéaires le long des trois axes (X, Y et Z). Il permet ainsi de détecter les mouvements, les chocs, et d’estimer l’orientation d’un objet (comme l'inclinaison).

\subsection{Gyroscope}
Le gyroscope est un capteur qui mesure la vitesse angulaire, c’est-à-dire la rapidité avec laquelle un objet tourne autour de ses axes (X, Y ou Z). Contrairement à l’accéléromètre qui détecte des mouvements linéaires, le gyroscope mesure les rotations. Le principe de fonctionnement du gyroscope repose sur la conservation du moment cinétique. Lorsqu’un rotor (ou masse tournante) est mis en rotation, il tend à conserver son orientation dans l’espace, conformément à la loi :\\
\begin{adjustwidth}{6cm}{0cm}$\dfrac{d\vec{L}_O}{dt} = 0 \Rightarrow \vec{L}_O = \vec{\text{Cte}}$
\end{adjustwidth}
\vspace{0.1cm}\\
avec $\vec{L}_O$ le moment cinétique par rapport à un point O.
\vspace{0.5cm}\\
Dans les gyroscopes MEMS (Micro-Electro-Mechanical Systems) modernes, cette rotation est généralement mesurée par effet Coriolis. Une petite structure oscillante à l’intérieur du capteur subit une déviation lorsqu’elle est en mouvement de rotation. Cette déviation est proportionnelle à la vitesse angulaire et peut être détectée électriquement.
\vspace{0.5cm}\\
Grâce à ces mesures, le gyroscope est capable de suivre les changements d’orientation d’un appareil, ce qui le rend essentiel pour notre processus de développement.

\subsection{Magnétomètre}
Le magnétomètre est un capteur qui mesure l’intensité et la direction d’un champ magnétique, généralement celui de la Terre. Il fonctionne comme une boussole numérique, permettant de déterminer l’orientation absolue d’un objet par rapport au nord magnétique. Contrairement au gyroscope, qui mesure des vitesses de rotation relatives, le magnétomètre fournit un repère fixe dans l’espace.
\vspace{0.5cm}\\
Le principe de fonctionnement le plus courant repose sur l’effet Hall. Lorsqu’un courant électrique traverse un conducteur ou un semi-conducteur soumis à un champ magnétique perpendiculaire, une différence de potentiel apparaît sur les côtés du matériau. Cette tension, appelée tension de Hall, est proportionnelle à l’intensité du champ magnétique traversant le capteur. Elle est mesurée électroniquement pour en déduire la composante du champ magnétique selon chaque axe (X, Y, Z).
\vspace{0.5cm}\\
Pour cette raison, il est souvent combiné avec un gyroscope et un accéléromètre, permettant une estimation plus fiable et précise de l’orientation dans l’espace.


\subsection{LiDAR}
Le LiDAR (Light Detection and Ranging) est un capteur qui mesure la distance entre un capteur et un objet en utilisant des impulsions lumineuses, généralement des lasers. Il permet de cartographier avec précision l’environnement en trois dimensions en mesurant le temps mis par la lumière pour revenir après avoir été réfléchie par un objet.
\vspace{0.5cm}\\
Le principe de fonctionnement repose sur la mesure du temps de vol (Time of Flight) d’un faisceau laser. Le capteur émet une impulsion lumineuse, puis détecte le retour de celle-ci après réflexion. En connaissant la vitesse de la lumière, il est possible de calculer la distance avec une grande précision. Cette méthode permet d’obtenir un nuage de points représentant la géométrie de l’environnement.
\vspace{0.5cm}\\
Grâce à sa capacité à fournir des mesures précises, même dans des conditions de faible luminosité, le LiDAR est largement utilisé en robotique, en cartographie et dans les véhicules autonomes pour la détection d’obstacles et la navigation.


\newpage
\section{Avantages et inconvénients}
Chaque capteur mesure une grandeur physique, mais aucun capteur n'est parfait. \\
Chacun présente des atouts et des limitations qu'il convient de connaitre afin de concevoir un système de mesure fiable.
\begin{itemize}
    \item \textbf{GPS} : \begin{itemize}
        \item \textit{Avantages} : \\ Le GPS a une bonne précision géographique (de l'ordre de quelques mètres pour un GPS civil).\\ De plus, il fonctionne de manière autonome. En effet, il n'a pas
        besoin d'infrastructure locale. En outre, il permet de déterminer la position absolue sur le globe (latitude, longitude, altitude).
        \item \textit{Limites} : \\ Même si le GPS nous offre plusieurs avantages. Ce dernier a quelques limites à ne pas ignorer. En fait, il a une latence élevée de l'ordre de plusieurs centaines de millisecondes à quelques secondes.\\
        Le GPS a aussi une faible fréquence d'échantillonnage (entre 1Hz et 10Hz), ce qui n'est parfois pas suffisant surtout pour capter les dynamiques rapides. Il ne faut pas oublier aussi qu'il est peu fiable
        en environnements clos comme les bâtiments, tunnels, etc... car il nécessite une ligne de vue avec les satellites. Finalement, le GPS est vulnérable aux interférences radio et au brouillage.
    \end{itemize}
    \item \textbf{Accéléromètre} \begin{itemize}
        \item \textit{Avantages} : \\ L'accélèromètre nous offre une mesure directe de l'accélération linéaire dans les 3 axes (x,y,z). En effet, ce dernier est très utile pour détecter les chocs, les mouvements  ou les vibrations.
        Sa haute fréquence d'échantillonnage (plusieurs KHz) lui permet d'avoir un suivi précis des changements rapides. Les avantages les plus importants de l'accélèromètre sont sa taille et son coût. En effet, il est peu coûteux et très compact.

        \item \textit{Limites} : \\ Tout comme le GPS, l'accélèromètre a ses inconvénients.En fait, un accélèromètre est très sensible au bruit (bruit thermique, vibrations parasites).
        Les mesures nécessitent souvent un filtrage notamment : un filtre passe-bas ou filtre de Kalman. De plus, un accélèromètre ne donne pas la position absolue. Il donne seulement des informations relatives au mouvement.

    \end{itemize}
    \item \textbf{Gyroscope} : 
    \begin{itemize}
        \item \textit{Avantages} : \\ Le gyroscope permet de mesurer la vitesse angulaire, c’est-à-dire les rotations autour des axes. \\ Il est très réactif et précis à court terme, ce qui le rend particulièrement utile pour détecter des changements rapides d’orientation. \\ Grâce à sa haute fréquence d’échantillonnage, il est capable de suivre les mouvements avec une grande précision en temps réel.
        
        \item \textit{Limites} : \\ Cependant, le gyroscope présente aussi certaines limites. \\ Il souffre notamment d’un phénomène appelé dérive : les petites erreurs de mesure s’accumulent avec le temps, ce qui dégrade la précision à long terme si le capteur n’est pas recalibré. \\ De plus, il consomme généralement plus d’énergie que d’autres capteurs comme l’accéléromètre, ce qui peut poser problème dans les systèmes autonomes. 
    \end{itemize}
    
    \item \textbf{Magnétomètre} : 
    \begin{itemize}
        \item \textit{Avantages} : \\ Le magnétomètre mesure le champ magnétique terrestre, ce qui permet d’estimer l’orientation absolue par rapport au nord magnétique. \\ Contrairement au gyroscope, il ne dérive pas avec le temps. \\ Il est donc particulièrement utile pour corriger les erreurs d’orientation dans les systèmes de navigation. 
        
        \item \textit{Limites} : \\ Le principal inconvénient du magnétomètre est sa grande sensibilité aux perturbations magnétiques locales. \\ En présence de métaux ferromagnétiques ou de champs électromagnétiques générés par des appareils électroniques, ses mesures peuvent devenir très imprécises. \\ Il nécessite souvent un étalonnage régulier pour fonctionner correctement, notamment en environnement intérieur ou urbain dense.
    \end{itemize}

    \item \textbf{LiDAR} :  
\begin{itemize}
    \item \textit{Avantages} : \\ 
    Le LiDAR permet de mesurer avec une grande précision la distance aux objets en utilisant des impulsions lumineuses. \\ 
    Il génère un nuage de points 3D détaillé, utile pour la cartographie, la détection d’obstacles et la modélisation de l’environnement. \\ 
    Contrairement aux caméras, il fonctionne efficacement dans des conditions de faible luminosité ou d’obscurité totale.

    \item \textit{Limites} : \\ 
    Le LiDAR peut être sensible aux conditions atmosphériques (pluie, brouillard, poussière) qui affectent la propagation des impulsions lumineuses. \\ 
    Il peut également être coûteux, énergivore, et sa portée est limitée par la puissance du faisceau et les propriétés réfléchissantes des surfaces. \\ 
    Les surfaces transparentes ou très absorbantes peuvent entraîner des erreurs ou des absences de mesure.
\end{itemize}


\end{itemize}

\newpage
\section{Sources de bruitages et/ou défauts techniques}
Les capteurs sont sujets à différents types de bruits :
\item \textbf{Sources de bruit des capteurs} :
\begin{itemize}
    \item \textit{GPS} : \ Bien que largement utilisé, le GPS reste sensible à divers bruits et erreurs. Les conditions atmosphériques, comme les variations dans la troposphère et l’ionosphère, peuvent déformer les signaux satellites, entraînant des erreurs de position. En zone urbaine ou montagneuse, les obstacles bloquent les signaux ou limitent la visibilité des satellites, réduisant la précision. De plus, le phénomène de multipath, où les signaux rebondissent avant d’atteindre le récepteur, fausse le calcul des distances.

    \item \textit{Accéléromètre et gyroscope} : \
Les capteurs inertiels souffrent d’erreurs internes dues à leur structure physique et électronique. Le \textit{bruit thermique}, causé par l’agitation des électrons, génère des fluctuations aléatoires. Les \textit{biais}, erreurs constantes influencées par la température ou l’usure, s’accumulent lors de l’intégration des mesures, entraînant une \textit{dérive} : le système détecte un mouvement inexistant. S’y ajoutent les erreurs d’alignement et les vibrations mécaniques, rendant ces capteurs peu fiables seuls sur le long terme sans recalibrage ou fusion de données.

    \item \textit{Magnétomètre} : \
Ce capteur mesure le champ magnétique terrestre, utile pour estimer l’orientation absolue (nord-sud). Cependant, sa fiabilité dépend fortement de l’environnement. Les objets \textit{ferromagnétiques} et les sources \textit{électromagnétiques} (moteurs, câbles, etc.) peuvent perturber la mesure. En intérieur, ces interférences sont fréquentes. Un \textit{étalonnage} régulier, par mouvements circulaires, est nécessaire pour corriger ces erreurs. Malgré cela, le magnétomètre reste peu fiable seul pour l’orientation.

\item \textit{LiDAR} : \\

Le LiDAR mesure la distance aux objets en envoyant des impulsions laser et en analysant leur temps de vol. Il fournit une représentation 3D précise de l’environnement. Cependant, plusieurs sources de bruit ou défauts techniques peuvent affecter la qualité des mesures. 

Les conditions atmosphériques telles que le brouillard, la pluie ou la poussière dispersent ou absorbent partiellement le faisceau laser, réduisant la portée effective et la précision du retour. Les surfaces très réfléchissantes (comme le verre ou les miroirs) peuvent provoquer des réflexions multiples, entraînant des mesures erronées ou décalées. À l’inverse, les matériaux sombres ou absorbants (comme les tissus noirs ou certaines peintures mates) peuvent réfléchir très peu de lumière, générant des données incomplètes ou du bruit.

Des erreurs peuvent aussi survenir à cause de la géométrie de la scène : des angles trop inclinés par rapport au capteur réduisent la puissance du signal réfléchi. De plus, les vibrations du support ou des erreurs de synchronisation dans les systèmes multi-capteurs peuvent introduire du flou ou des artefacts dans le nuage de points. Enfin, les LiDARs à balayage mécanique peuvent s’user ou se désaligner avec le temps, affectant leur précision.

Malgré ces limitations, le LiDAR reste l’un des capteurs les plus fiables pour la perception 3D, notamment lorsqu’il est utilisé en complément d’autres capteurs.

\end{itemize}


