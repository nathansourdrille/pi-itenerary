\section{Filtrage des signaux bruts}

Le filtrage des données brutes consiste à effectuer un prétraitement afin d'améliorer la qualité des signaux issus des capteurs. L'objectif principal est de garantir la cohérence des acquisitions et de supprimer les valeurs aberrantes (outliers) susceptibles de fausser les analyses ou traitements ultérieurs.

\subsection{Synchronisation des signaux}

Avant tout traitement, nous avons procédé à la synchronisation temporelle des différentes acquisitions issues des capteurs (gyroscope, accéléromètre, magnétomètre, etc.). Cela permet de garantir que les mesures analysées correspondent aux mêmes instants physiques, ce qui est essentiel pour toute fusion ou comparaison de données multisources. La synchronisation a été réalisée en alignant les horodatages fournis par chaque capteur.

\subsection{Détection des valeurs aberrantes}

Une fois les données synchronisées, nous avons utilisé des représentations graphiques de type \textbf{boxplot} (ou boîtes à moustaches) pour visualiser la distribution des données de chaque capteur. Ces graphiques permettent d’identifier facilement les éventuelles valeurs aberrantes, c’est-à-dire les points très éloignés de la médiane ou dépassant l’intervalle interquartile étendu (défini en cours par $[Q1 - 1{,}5 \times IQR,\ Q3 + 1{,}5 \times IQR]$).

Ces valeurs peuvent provenir de bruits ponctuels, de pertes de signal ou d'erreurs transitoires dans la transmission. Elles ont été exclues des jeux de données à l'aide d’un filtre conditionnel simple basé sur ces bornes statistiques.

\subsection{Lissage et filtrage complémentaire}

En complément, un filtrage numérique a pu être appliqué sur certaines séries temporelles à fort bruit, notamment via un filtre passe-bas de type moyenne glissante (moving average) ou un filtre de Savitzky-Golay, selon le type de signal et le niveau de bruit observé. Cela permet de réduire les fluctuations rapides non représentatives sans altérer les tendances générales du mouvement.

\subsection{Résultat du prétraitement}

Après filtrage, les signaux présentent une meilleure continuité et une diminution significative du bruit. Les valeurs extrêmes ont été supprimées, rendant les signaux plus exploitables pour les étapes suivantes telles que la calibration dynamique ou la fusion multi-capteurs.

